%Preamble command definitions%
%---Author affiliation formatting-----
\newcommand\Mark[1]{\textsuperscript{#1}}

%---Typography-----
% \newcommand*{\eg}{e.\,g.\@\xspace}%
% \newcommand*{\ie}{i.\,e.\@\xspace}%
% In english typography, there is no space between 'e.' and 'g.' etc.
% See The Art of Typographic Style for further reading
\newcommand*{\eg}{e.g.\@\xspace}%
\newcommand*{\ie}{i.e.\@\xspace}%54
\newcommand*{\wrt}{w.r.t.\@\xspace}%54
\newcommand{\xmark}{\ding{55}}%
\newcommand{\cmark}{\ding{51}}%

%---In-text references to labels-----
\newcommand{\refsec}[1]{\hyperref[#1]{Section~\ref{sec:#1}}}
\newcommand{\refequ}[1]{\hyperref[#1]{Equation~\eqref{eq:#1}}}
\newcommand{\reffig}[1]{\hyperref[#1]{Figure~\ref{fig:#1}}}
\newcommand{\reftbl}[1]{\hyperref[#1]{Table~\ref{tbl:#1}}}
\newcommand{\refass}[1]{\hyperref[#1]{Assumption~\ref{ass:#1}}}
\newcommand{\refthm}[1]{\hyperref[#1]{Theorem~\ref{thm:#1}}}
\newcommand{\reflemma}[1]{\hyperref[#1]{Lemma~\ref{lemma:#1}}}
\newcommand{\refalgo}[1]{\hyperref[#1]{Algorithm~\ref{algo:#1}}}
\newcommand{\refappendix}[1]{\hyperref[#1]{\ref{appendix:#1}}} % "Appendix~"
                                % is in renewed \thesection after \appendix

%------------------------------------------------------------------------------%
% Color schemes for color-blind readers: red/green blindness%
% [TOL, Paul: Color Schemes. SRON/EPS/TN/09-002, 29 September 2018]%
% Muted qualitative color schemes%
\definecolor{CMQBlueA}{HTML}{332288}  %indigo%
\definecolor{CMQBlueB}{HTML}{88CCEE}  %cyan%
\definecolor{CMQBlueC}{HTML}{44AA99}  %teal%
\definecolor{CMQGreenA}{HTML}{117733} %green%
\definecolor{CMQGreenB}{HTML}{999933} %olive%
\definecolor{CMQGreenC}{HTML}{DDCC77} %sand%
\definecolor{CMQRedA}{HTML}{CC6677}   %rose%
\definecolor{CMQRedB}{HTML}{882255}   %wine%
\definecolor{CMQRedC}{HTML}{AA4499}   %purple%
\definecolor{CMQGray}{HTML}{DDDDDD}   %pale gray%
%
%---R-Plot related-------------------------------------------------------%
\renewcommand{\rothead}[2][90]{\makebox[9mm][c]{\rotatebox{#1}{\makecell[c]{#2}}}}%
% ---Tiks/Pgfplot related-------------------------------------------------------%
\newcommand{\captionGlo}   {}%
\newcommand{\captionLocL}  {}%
\newcommand{\captionLocR}  {}%
\newcommand{\labelGlo}     {}%
\newcommand{\labelLocL}    {}%
\newcommand{\labelLocR}    {}%
\newcommand{\inputFigLocL} {}%
\newcommand{\inputFigLocR} {}%
%
\renewcommand\mycommfont[1]{\footnotesize\ttfamily\textcolor{blue}{#1}}
\SetCommentSty{mycommfont}
%
\newcommand*{\figureTwoColumn}[2]{%
	\ifdefined\varInputTable
		\pgfplotsset{width=\linewidth, height=2.5cm, compat=1.15}
		\pgfplotsset{legend style={font=\footnotesize}}
		\def \msp {0.0cm}%
		\begin{figure*}[ht]%
			\centering%
			\setlength{\abovecaptionskip}{1em}%
			\setlength{\belowcaptionskip}{0pt}%
			\parbox[][][s]{0.49\linewidth}{%
				\centering%
				\subcaptionbox{\captionLocL\labelLocL}{%
					\input{#1}%
				}%
			}%
			\hspace*{\msp}%
			\begin{minipage}{0.49\linewidth}%
				\centering%
				\subcaptionbox{\captionLocR\labelLocR}{%
					\input{#2}%
				}%
			\end{minipage}%
			\caption{\captionGlo}%
			\labelGlo%
		\end{figure*}%
	\else%
	\fi%
}%

\newcommand*{\figureTwoColumnB}[2]{%
	\ifdefined\varInputTable
	\pgfplotsset{width=\linewidth, height=2.5cm, compat=1.15}
	\pgfplotsset{legend style={font=\footnotesize}}
	\def \msp {0.0cm}%
	\begin{figure*}[!h]%
		\centering%
		\setlength{\abovecaptionskip}{1em}%
		\setlength{\belowcaptionskip}{0pt}%
		\parbox[][][s]{0.49\linewidth}{%
			\centering%
			\subcaptionbox{\captionLocL\labelLocL}{%
				\input{#1}%
			}%
		}%
		\hspace*{\msp}%
		\begin{minipage}{0.49\linewidth}%
			\centering%
			\subcaptionbox{\captionLocR\labelLocR}{%
				\input{#2}%
			}%
		\end{minipage}%
		\caption{\captionGlo}%
		\labelGlo%
	\end{figure*}%
	\else%
	\fi%
}%


\newcommand{\tableTwoColumn}[1]{%
	\ifdefined\varInputTable
	\begin{table*}[ht]%
		\footnotesize%commet out in ieee style mode
		\begin{center}%
			\caption{\captionGlo}%
			\renewcommand{\arraystretch}{1.3}%
			\input{ts_tables/#1}%
			\labelGlo%
		\end{center}%
	\end{table*}%
	\else\fi%
}%


\newcommand{\tableTwoColumnH}[1]{%
	\ifdefined\varInputTable
	\begin{table*}[h]%
		\footnotesize%commet out in ieee style mode
		\begin{center}%
			\caption{\captionGlo}%
			\renewcommand{\arraystretch}{1.3}%
			\input{ts_tables/#1}%
			\labelGlo%
		\end{center}%
	\end{table*}%
	\else\fi%
      }%
      
\newcommand{\tableTwoColumnF}[1]{%
	\ifdefined\varInputTable
	\begin{table*}[htbp]%
		\footnotesize%commet out in ieee style mode
		\begin{center}%
			\caption{\captionGlo}%
			\renewcommand{\arraystretch}{1.3}%
			\input{ts_tables/#1}%
			\labelGlo%
		\end{center}%
	\end{table*}%
	\else\fi%
}%

%TabularX Versions of l c r columns
\newcolumntype{L}[1]{>{\hsize=#1\hsize\raggedright\arraybackslash}X}%
\newcolumntype{R}[1]{>{\hsize=#1\hsize\raggedleft\arraybackslash}X}%
\newcolumntype{C}[1]{>{\hsize=#1\hsize\centering\arraybackslash}X}%

\section{Measuring money in circulation}
\label{sec:measure}%

In \cite{pernice2019cryptocurrencies} monetary units are seen as ``in circulation'' for day \(\perd\)
% covering \([\perd_{\Start},\perd_{\End}]\)
if they have been used as a medium of exchange. %
Money is referred to as circulating if it has been moved economically within the last day, month, year or generally any time period $\wndw$ covering $[\wndw_\Start , \wndw_\End]$. %
This period can be described relatively by its length \(\wndwLength = \wndw_\End-\wndw_\Start\) requiring that \(\wndw_\End = \perd_{\End}\). %
The choice of setting \(\wndw_\End\) equal to \(\perd_{\End}\)  makes measurement-day \(\perd\) overlap with the window \(\wndw\) for which money in circulation is to be estimated (compare figure~\ref{fig:timevars}). %
Monetary units brought into circulation in the last second of day \(\perd\) are to be attributed to belong to money in circulation for \(\perd\). %  

\begin{figure*}[ht!]%
	\centering
	\ifdefined\varInputFigs%
%	\includegraphics[width=0.8\linewidth]{fig/mcirc_concept_window_uneqal_period_HR}%
	\input{ts_figs/mcirc_concept}
	\else%
	\fi%
	\caption{%
		An example of a transaction chain. %
	}%
	\label{fig:mcirc_concept}%
\end{figure*}%


\subsection{Intuition and original approach}
\label{sec:orig_approach}%
To estimate the subset of money supply which circulated within $\wndw$, the authors suggested to analyze the transactions recorded window $\wndw$.  %
Ignoring change, outputs of coinbase transactions and as well as transactions spending outputs generated before $\wndw$ are interpreted as bringing an amount into circulation that corresponds to the value of spent outputs.  %
All inputs referring to UTXOs generated \emph{within} period $\wndw$, on the
other hand, re-spend money which has already been counted as circulating.%
%
\textbf{[@TODO Check against illustration.]}
This can be illustrated with \reffig{mcirc_concept} where values are symbolized with coins. %
Here, we would need to determine, how many monetary units have made the transaction volume $8$ (sum of outputs in C, D, E and F excluding self-churn) during period $\wndw$ possible. %
In this example, we would need to focus on transactions A, B and D in contrast to transactions C and E which reused monetary unspent transaction outputs that have been generated within period $\wndw$. %

If change in transactions is to be considered, the general approach is complicated by a technicality of UTXO-based cryptocurrencies: Transactions always spend prior transaction outputs in full. %

The authors defined two general approaches: \ac{wba} and \ac{mca}.  %
\ac{wba} considers the sum of all inputs \textit{in circulation} (as technically all has been proven to be available for transactions) while the \ac{mca} only counts the fraction economically sent to third parties.  %
%
They are visualized in \reffig{mcirc_concept}.  %
The moved-coin approach considers only output $\mathsf{Out_1}$ of transaction $\mathsf{Tx_C}$ as circulating, not the change output.  %
This approach captures the net economic value transferred to a third party.  %
The \ac{wba} classifies the whole input of transaction $\mathsf{Tx_C}$ as circulating.  %

However, as for UTXO blockchains the relation between transaction inputs and outputs are not determined, \ac{mca} might not be defined clearly. %
If for a given transaction one input was generated within and one before $\wndw$, it remains unspecified which one corresponds to the change output.  %

The authors thus utilized assignment rule between transaction inputs and outputs utilizing the terminology of cost accounting: %
They differentiate between \ac{fifo}, where youngest inputs get assigned to outputs first, and \ac{lifo}, where it is the other way around.  %

The authors introduced three definitions of money in circulation as measured for day \(\perd\), each defined by window length $\wndwLength$: %
Money in circulation for period $\wndw$ adopting the \ac{wba} ($ \MCircWbPWl $), and both the \ac{mca} with the \ac{lifo} rule ($ \MCircMlPWl $) and the \ac{fifo} rule ($ \MCircMfPWl $).

\subsection{Necessity for a more efficient implementation}
\label{sec:necessity_effi}%
While analyzing data for money which has been circulating within the last day as in \cite{pernice2019cryptocurrencies} is interesting, the more interesting case is probably basing money in circulation on a monthly or yearly basis. %
The implementation suggested by the authors, however, is rather inefficient for larger look-back time windows. %

The algorithm proposed in \cite{pernice2019cryptocurrencies} loops over-and-over again over the same transactions: %
As illustrated in the conceptional description of the three velocity measures, determining whether an unspent output was spent in a given time window is essential for the chosen approach. %
While this is computationally feasible for small time windows, already for look-back windows of a few days, naively looping through all transactions is problematic. %
For daily timeseries data and multiday look-back time windows the data for computations overlaps. %
The published code on \url{https://github.com/wiberlin/ccurr_velocity} for a look-back time window of a week takes XYZ hours to terminate even if restricted on the first 3 years of the Bitcoin blockchain and run using all XY cores of a server with XY GB ram. 
\textbf{[TODO George: Here comes some O-Notation magic!]}
We thus propose an improved algorithmic implementation proceeding inductively. %


\begin{figure*}
  \centerline{%
    \ifdefined\varInputFigs%
    \includegraphics[width=0.8\linewidth]{fig/period_variables}%
    \else%
    \fi%
  }%
  \caption{Illustration of time-related variables with \(\wndw\) relating to the time windows characterizing the definition of money in circulation as measured for day \(\perd\) (compare section \ref{sec:measure}) and \(\iterp\) relating to the period covered by each iteration step of the novel inductive implementation (compare section \ref{sec:novel_impl}).}
  \label{fig:timevars}
\end{figure*}
%

\subsection{Inaccuracy due to mistreatment of transaction fees}
\label{sec:inaccuracy_fees}%
Relying only on non-recycling transaction outputs and coinbase transactions to estimate money in circulation, the approach chosen by \cite{pernice2019cryptocurrencies} misrepresents fees as always in circulation as they are captured by outputs of coinbase transactions. %
This is a slight misrepresentation. %
Fees of transactions that have been respent in within time window \(\wndw\) ought to be considered respent as well. %
We thus, in addition to addressing the issue summarized in section \ref{sec:necessity_effi}, suggest a slight adaptation improving the estimates precision. %

\subsection{Disentanglement of  transaction fees}
\label{sec:improve_fees}%


\subsection{Efficiency improvements}
\label{sec:novel_impl}%

\subsubsection{Intuition}
Our approach allows for reusing and adjusting information of calculations for prior time-windows instead of recalculating them entirely for every day. %
While the novel implementation allows effectively for larger time-windows \(\wndwLength\), we reuse the general approach for measuring money in circulation proposed in \cite{pernice2019cryptocurrencies}. %
We do adapt it, however, by adjusting and recombining its elements. %
Roughly spoken, a single inductive step takes the measure for money in circulation \( \MSetCircAggr \) for time window \(\wndw\) and then: %
\begin{enumerate}
\item Shifts window \(\wndw\) so that it starts now at \(\wndw_{\Start} + 1\) and ends at \(\wndw_{\End} + 1\),
\item adds the value of monetary units which are ``activated'' by the transactions of the time period covered by the period \(\wndw_{\End}\) to \(\wndw_{\End} + 1\) and thus \(\iterp\) and %
\item subtract monetary units that are no longer part of the new time window. %
\end{enumerate}
Practically, we are calculating and precashing daily helper-constructs to later use them in an inductive aggregation process. %  
We will thus introduce the calculation of helper-structures first and only then turn to a detailed explanation of the induction algorithm. %

\subsubsection{Counting coins brought into active circulation}

The aforementioned helper-structure might be seen as generalization of the estimator for money in circulation introduced in \cite{pernice2019cryptocurrencies}: %
Assume that money in circulation with a look-back time window of length \(\wndwLength\) is to be estimated for day \(\iterp\). %
We thus estimate how much money can been brought into active circulation during the period \([\iterp_{\Start},\iterp_{\End}]\): Before the events at day \(\iterp\), the respective money supply has remained untouched for \(\wndwLength\) days. With its transaction at day \(\iterp\), however, it can now be seen as circulating within period  \([\wndw_\Start\) to \(\wndw_\End]\). %
While the helper-structure is estimating the value of monetary units which are ``activated'' by transactions of day \( \iterp \), it does neglect outputs which turn into ``already spent'' outputs in prior days within window \(\wndw\). %
As a consequence it only coincides with the original estimator for the special case of \(\iterp = \wndw\). %
While the original estimator requires the information of each transaction in time-window \(\wndw\), the helper-construct looks only at the transactions of iteration-step \(\iterp\) and checks their inputs age against \(\wndw_\Start\). % 

The above helper-construct can be seen as the centerpiece to realize the inductive framework. %
Algorithm \ref{algo:code_mcirc} offers a detailed summary. %

For every period $\iterp$, we loop over all transactions $t\in\TxP$ and add
their inputs to circulating money if they either reference outputs from
coinbase transactions, denoted by $\genByCoinbase(i)$, or outputs with
timestamps $\dateGen(i)$ before the first timestamp $\wndw_\Start$ of period
$\wndw$.  %
%
%  \footnote{In \reffig{mcirc_concept}, the algorithm would go trough %
%  transaction C (adding output $\mathsf{1}$ of transaction B, referenced by input $\mathsf{1}$ %
%  of transaction C), would then go to transaction D (finding no input at all), %
%  then to transaction E (skipping output $\mathsf{1}$ of transaction E but adding output $\mathsf{1}$ %
%  of transaction D as referenced by input $\mathsf{2}$ of transaction E) and so forth. %
% } %
%
%\ifdefined\varInputAlgos%
\renewcommand{\arraystretch}{1.5}%
\begin{algorithm*}[!h]%
	\DontPrintSemicolon
	\caption{Whole-bill-approach: Measurement of $\protect\MCircWb$ for period $\perd$ with look-back window $\wndw$.}\label{algo:code_mcirc_wb}%
	\KwData{$\wndw_\Start$                                    	    \tcc*{Beginn of look-back window $w$}}%
	\KwData{$\TxW$                                              	\tcc*{Set of transactions in period $\wndw$}}%
%	\KwResult{$\MCircWb$                                           \tcc*{Money supply to be estimated within period $\wndw$}}%
%
	$ \MCircWb \gets {0} $											\tcc*{Money supply to be estimated within period $\wndw$}%
%
	\ForEach {$ t \in \TxW $                                      	\tcc*[]{Loop through transactions $t$ of period $\wndw$}}%
	{%
		\ForEach {$ i \in \InpT $                                 	\tcc*[]{Loop through inputs $i$ of transaction $t$}}%
		{%
			\If{%
				$\dateGen(i)<\wndw_\Start$\ or\ $\genByCoinbase(i)$\label{algo:code_mcirc_wbCond}\tcc*{Was $i$ generated before $\wndw_\Start$/from a coinbase transaction?}%
			}{%
				$\MCircWb \gets{\MCircWb + \valI(i)}$				\tcc*{Update money supply}%
			}%
		}%
	}%
	return $\MCircWb$                                             	\tcc*{Result: Return money in circulation within $\wndw$ as estimated for period \(\perd\) for \ac*{wba}}%
\end{algorithm*}%
\else%
\fi%
%



% \refalgo{code_mcirc_wb}.  %
% For every period $\wndw$, we loop over all transactions $t\in\TxW$ and add
% their inputs to circulating money if they either reference outputs from
% coinbase transactions, denoted by $\genByCoinbase(i)$, or outputs with
% timestamps $\dateGen(i)$ before the first timestamp $\wndw_\Start$ of period
% $\wndw$.  %
% %
% %  \footnote{In \reffig{mcirc_concept}, the algorithm would go trough %
% %  transaction C (adding output $\mathsf{1}$ of transaction B, referenced by input $\mathsf{1}$ %
% %  of transaction C), would then go to transaction D (finding no input at all), %
% %  then to transaction E (skipping output $\mathsf{1}$ of transaction E but adding output $\mathsf{1}$ %
% %  of transaction D as referenced by input $\mathsf{2}$ of transaction E) and so forth. %
% % } %
% %
% %\ifdefined\varInputAlgos%
\renewcommand{\arraystretch}{1.5}%
\begin{algorithm*}[!h]%
	\DontPrintSemicolon
	\caption{Whole-bill-approach: Measurement of $\protect\MCircWb$ for period $\perd$ with look-back window $\wndw$.}\label{algo:code_mcirc_wb}%
	\KwData{$\wndw_\Start$                                    	    \tcc*{Beginn of look-back window $w$}}%
	\KwData{$\TxW$                                              	\tcc*{Set of transactions in period $\wndw$}}%
%	\KwResult{$\MCircWb$                                           \tcc*{Money supply to be estimated within period $\wndw$}}%
%
	$ \MCircWb \gets {0} $											\tcc*{Money supply to be estimated within period $\wndw$}%
%
	\ForEach {$ t \in \TxW $                                      	\tcc*[]{Loop through transactions $t$ of period $\wndw$}}%
	{%
		\ForEach {$ i \in \InpT $                                 	\tcc*[]{Loop through inputs $i$ of transaction $t$}}%
		{%
			\If{%
				$\dateGen(i)<\wndw_\Start$\ or\ $\genByCoinbase(i)$\label{algo:code_mcirc_wbCond}\tcc*{Was $i$ generated before $\wndw_\Start$/from a coinbase transaction?}%
			}{%
				$\MCircWb \gets{\MCircWb + \valI(i)}$				\tcc*{Update money supply}%
			}%
		}%
	}%
	return $\MCircWb$                                             	\tcc*{Result: Return money in circulation within $\wndw$ as estimated for period \(\perd\) for \ac*{wba}}%
\end{algorithm*}%
\else%
\fi%
%

% Measuring money in circulation under the \ac{mca} is depicted in
% \refalgo{code_mcirc_mc}.  %
% %
% As in \refalgo{code_mcirc_wb}, for time window $\wndw$ we loop over all
% transactions $t\in\TxP$ and add inputs based on the same core condition
% (compare lines~\ref{algo:code_mcirc_mc}-\ref{algo:code_mcirc_mcCond} and
% \ref{algo:code_mcirc_wb}-\ref{algo:code_mcirc_wbCond}).  %
% This time, however, only those inputs are regarded for further counting,
% which add up to the amount sent to third parties. Therefore, the calculated
% amount of money in circulation per transaction can be less or equal to the
% amount sent to third parties, but never more.
% The order of inputs to consider is determined by the \ac{lifo} or \ac{fifo}
% principle.  %
% For every transaction $t$ in time window $\wndw$, the amount sent to third
% parties is determined net of self-churn as
% $\valO^\sendToOthers(t) = \sum_{o\in{}\Out'_{t}} \valO(o)$ %
% where $\Out'_{t}$ denotes non-self-churn outputs of transaction $t$.  %
% If all outputs are identified as self-churn, $\valO^\sendToOthers(t) = 0$ and
% the algorithm continues with the next transaction.  %
% %
% If $\valO^\sendToOthers(t) > 0$, the algorithm collects input values in a
% vector $\InpSortedT$; they are sorted in either ascending (\ac{lifo}) or
% descending (\ac{fifo}) order \wrt to the timestamp when the UTXOs were
% generated.  %
% Then, looping over inputs $i$, input values $\valI(i)$ are added to
% $\MCircMPWl(t)$ if they meet the core condition (compare line 17) introduced
% in line 6 of \refalgo{code_mcirc_wb}.  %
% \footnote{Summands $\MCircMPWl(t)$ can be interpreted as money drawn into
%   effective circulation by transaction $t$.} %
% However, one additional condition applies: If the last added input would
% increase the summand $\MCircMPWl(t)$ beyond the value of outputs sent to
% third parties $\valO^\sendToOthers(t)$, we only add up to the latter
% amount.  %
% % This effectively adds only the necessary fraction of the inputs of
% % transaction $t$, generated before window $w$ or from coinbase transactions,
% % that were required to match $\valO^\sendToOthers(t)$.  %

%   % The components $\MCircMPWl(t)$ are consecutively summing all transaction
%   % values to calculate money in circulation $\MCircMPWl$ for the given period
%   % $p$ with respect to time window $\wndw$ as well as the applied sorting type.

% % \par While our algorithms theoretically might be applied to any window length $\wndwLength$, in practice they lack performance for large windows. %
% % Additional methods of optimization might be needed, to apply the method windows larger then 30 days. %


% 
%%% Local Variables:
%%% mode: latex
%%% TeX-master: "../main"
%%% End:
%

\begin{figure}[t]
	%-------------------------------------------------------------------------------
\usetikzlibrary{calc}
%coordinate test point. Use as follows: 	\coordinate (cen) 	 	 at (0,0) (cen) [point];
\tikzset{
	every point/.style = {radius={\pgflinewidth}, opacity=1, draw, solid, fill=white},
	pt/.pic = {
		\begin{pgfonlayer}{foreground}
			\path[every point, #1] circle;
		\end{pgfonlayer}
	},
	point/.style={insert path={pic{pt={#1}}}}, point/.default={},
	point name/.style = {insert path={coordinate (#1)}}
}
%-------------------------------------------------------------------------------

% basic delacations
\pgfdeclarelayer{background}%
\pgfdeclarelayer{foreground}%
\pgfsetlayers{background,main,foreground}%

%Coin Symbol--------------------------------------------------------------------
\tikzset{%
	pics/coin/.style n args={6}{%
		code ={
			\def \coinLineWidth {#1}%
			\def \coinCenDist   {#2}%
			\def \circRad       {#3}%
			\def \circColor     {#4}%
			\def \coinColorL    {#5}%
			\def \coinColor     {#6}%
			\def \signRad       {\circRad*0.20}%
			\def \signRadProp   {0.45}
			\def \signRadExt    {\signRad*\signRadProp}
			\def \signRadOut    {\signRad+\signRadExt}
			\def \signAngle     {30}
			\def \sinSignAngle  {sin(\signAngle)}
			\def \cosSignAngle  {cos(\signAngle)}
			\def \signWidth     {{pow((pow(\signRadOut,2)-pow(\sinSignAngle*\signRad,2)),0.5)-(\cosSignAngle)*\signRad)}}
			
			\def \startIX       {{cos(\signAngle)*\signRad}}
			\def \startIY       {{\sinSignAngle*\signRad}}
			
			\def \startOY       {\startIY}
			\def \signAngleOutS {{atan((\sinSignAngle*\signRad)/pow((pow(\signRadOut,2)-pow(\sinSignAngle*\signRad,2)),0.5))}}
			\def \signAngleOutE {{360-atan((\sinSignAngle*\signRad)/pow((pow(\signRadOut,2)-pow(\sinSignAngle*\signRad,2)),0.5))}}
	%
			\coordinate ()          at ( 0.0     , 0.0);%
			\coordinate (cen)       at ( 0.0     , 0.0);%
			\coordinate (cenB)      at ($(cen)    + ( 0.00,-\coinCenDist)$);%
						
			\node[
				circle,
				draw = \coinColorL,
				fill = \circColor,
				line width = \coinLineWidth,
				minimum size = \circRad,
				inner sep = 0pt,
				outer sep = 0pt,
			](NCD) at (cenB) {};
			
			\node[
				circle,
				draw = \coinColorL,
				fill = \circColor,
				line width = \coinLineWidth,
				minimum size = \circRad,
				inner sep = 0pt,
				outer sep = 0pt,
			](NCU) at (cen) {};
			
%			\draw[ - , line width = \coinLineWidth, black] (NCU.195) -- (NCD.195);
			\draw[-, line width = \coinLineWidth, \coinColorL] (NCU.210) -- (NCD.210);
			\draw[-, line width = \coinLineWidth,\coinColorL] (NCU.225) -- (NCD.225);
			\draw[-, line width = \coinLineWidth, \coinColorL] (NCU.240) -- (NCD.240);
			\draw[-, line width = \coinLineWidth, \coinColorL] (NCU.255) -- (NCD.255);
			\draw[-, line width = \coinLineWidth, \coinColorL] (NCU.270) -- (NCD.270);
			\draw[-, line width = \coinLineWidth, \coinColorL] (NCU.285) -- (NCD.285);
			\draw[-, line width = \coinLineWidth, \coinColorL] (NCU.300) -- (NCD.300);
			\draw[-, line width = \coinLineWidth, \coinColorL] (NCU.315) -- (NCD.315);
			\draw[-, line width = \coinLineWidth, \coinColorL] (NCU.330) -- (NCD.330);
%			\draw[ - , line width = \coinLineWidth, black] (NCU.345) -- (NCD.345);
			
			\draw[
				 - ,
				 line width = \coinLineWidth,
				 \coinColorL,
				 fill = \coinColor
			]
				(\startIX,-\startIY) arc (360-\signAngle:\signAngle:\signRad)
				-- +(\signWidth, 0.0)
				arc (\signAngleOutS:\signAngleOutE:\signRadOut)
				-- (\startIX,-\startIY)
				-- cycle
				;
		}%
		
	} ,%
	pics/coin/.default={1.0pt}{0.1}{1cm}{bg}{black!80!white}{black!90!white}%
}%
%
%\only<1->{%
%\only<1-7>{%
\begin{tikzpicture}[scale=1.0]
	\def \debugPoint   {}%
%	\def \debugPoint   {point}%
	% font settings ############################################################
	\def \fontset      {\scriptsize\sffamily}
	% color definition #########################################################
	\def \colorOut     {jwigreige!50}
	\def \colorOutH    {jwigreige!25!black}
	\def \colorIn      {black!10!white}
	\def \colorCoin    {white}
	\def \colorCoinY   {jwilightgreen!80!bg}
	\def \colorCoinN   {jwiorange}
	\def \colorCoinUtxoL {black!30!white}
	\def \colorCoinUtxo  {black!40!white}
	% arrow and line definitions ###############################################
	\def \lineW        {1.3pt}
	% width/height definition ##################################################
	\def \picW         {\columnwidth*31/31}%
	\def \picH         {\paperheight*24/31}%
	\def \TH           {\picH*0.4}%picH*0.5*0.8
	
	\def \WBA          {\picW*11.50/31}
	\def \WLA          {\picW* 7.75/31}
	\def \WBB          {\picW* 4.00/31}
	\def \WLB          {\picW* 0.25/31}
	\def \WBC          {\picW* 3.50/31}
	\def \WBD          {\picW*10.50/31}
	\def \WLD          {\picW*14.25/31}
	% width/height nodes/boxes###################################################
	\def \BOH          {\picH* 0.35  }
	\def \BOW          {\picW* 6.0/31}
	
	% node positioning
	\def \BIPX         {\BOW*0.2375}
	
	% tikz styles################################################################
	\tikzstyle{lineIO}       = [
		- ,
		line width = \lineW*0.8,
	]
	\tikzstyle{lineION}      = [
		- ,
		line width = \lineW*0.8,
		dotted,
	]
	\tikzstyle{nodeOuter}    = [
		draw,
		rectangle,
		rounded corners=5pt,
		solid,
		line width = \lineW,
		black,
		fill=\colorOut,
		inner sep = 0pt,
		outer sep = 0pt,
		text width = \BOW,
		minimum width = \BOW,
		minimum height = \BOH*1.075,
	]
	\tikzstyle{nodeOuterH}   = [
		draw,
		rectangle,
		rounded corners=5pt,
		solid,
		line width = \lineW,
		black,
		fill=\colorOutH,
		inner sep = 0pt,
		outer sep = 0pt,
		text width = \BOW,
		minimum width = \BOW,
		minimum height = \BOH*1.075,
	]
	\tikzstyle{nodeInner}    = [
		draw,
		rectangle,
		rounded corners=3pt,
		solid,
		line width = \lineW*0.9,
		black,
		fill = \colorIn,
		inner sep = 0pt,
		outer sep = 0pt,
	]
	\tikzstyle{nodeInnerA}   = [
		nodeInner,
		align = center,
		text width = \BOW*0.9-3pt,
		minimum width = \BOW*0.9,
		minimum height = \BOH*0.15,
	]
	\tikzstyle{nodeInnerB}   = [
		nodeInner,
		align = left,
		text width = \BOW*0.425-5pt,
		minimum width = \BOW*0.425,
	]	
	\tikzstyle{nodeInnerBY}  = [
		nodeInner,
		align = left,
		text width = \BOW*0.425-5pt,
		minimum width = \BOW*0.425,
		fill = jwiblue!80!bg,
	]
	\tikzstyle{nodeInnerBA}  = [
		nodeInnerB,
		minimum height = \BOH*0.15,
	]		
	\tikzstyle{nodeInnerBAY} = [
		nodeInnerBY,
		minimum height = \BOH*0.15,
	]
	\tikzstyle{nodeInnerBAC} = [
		nodeInnerB,
		minimum height = \BOH*0.15,
		align = center,
	]
	\tikzstyle{nodeInnerBACY}= [
		nodeInnerBY,
		minimum height = \BOH*0.15,
		align = center,
	]
	\tikzstyle{nodeInnerBB}  = [
		nodeInnerB,
		minimum height = \BOH*0.25,
	]
	\tikzstyle{nodeInnerBBY} = [
		nodeInnerBY,
		minimum height = \BOH*0.25,
	]	
	\tikzstyle{nodeInnerBC}  = [
		nodeInnerB,
		minimum height = \BOH*0.35,
	]
	\tikzstyle{nodeInnerBCY} = [
		nodeInnerBY,
		minimum height = \BOH*0.35,
	]
	\tikzstyle{nodeInnerBD}  = [
		nodeInnerB,
		minimum height = \BOH*0.55,
	]
	\tikzstyle{nodeInnerBDY} = [
		nodeInnerBY,
		minimum height = \BOH*0.55,
	]
	
	% coordinates ###############################################################
	% 1 | 2 | 3 | 4 | 5 | 6 | 7 | 8 | 9 | 10 | 11 | 12 | 13 | 14 |
	% A | B | C | D | E | F | G | H | I |  J |  K |  L |  M |  N |
	
	\coordinate (cen)       at ($(0.0,0.0)   + ( 0.0       , 0.0         )$) (cen)   [\debugPoint];%
	
%	\coordinate (PLU)       at ($(cen)       + (-\picW*0.5 , \picH*0.5   )$) (PLU)   [\debugPoint];%
%	\coordinate (PLD)       at ($(cen)       + (-\picW*0.5 ,-\picH*0.5   )$) (PLD)   [\debugPoint];%
%	\coordinate (PRU)       at ($(cen)       + ( \picW*0.5 , \picH*0.5   )$) (PRU)   [\debugPoint];%
%	\coordinate (PRD)       at ($(cen)       + ( \picW*0.5 ,-\picH*0.5   )$) (PRD)   [\debugPoint];%
	
%	\coordinate (TC)        at ($(cen)       + ( 0.0       ,-\TH         )$) (TC)    [\debugPoint];%
%	\coordinate (TL)        at ($(TC)        + (-\picW*0.5 , 0.0         )$) (TL)    [\debugPoint];%
%	\coordinate (TR)        at ($(TC)        + ( \picW*0.5 , 0.0         )$) (TR)    [\debugPoint];%
%	
%	\coordinate (BAC)       at ($(cen)       + (-\WBA      , 0.0         )$) (BAC)   [\debugPoint];%
%	\coordinate (LAC)       at ($(cen)       + (-\WLA      , 0.0         )$) (LAC)   [\debugPoint];%
%	\coordinate (BBC)       at ($(cen)       + (-\WBB      , 0.0         )$) (BBC)   [\debugPoint];%
%	\coordinate (LBC)       at ($(cen)       + (-\WLB      , 0.0         )$) (LBC)   [\debugPoint];%
%	\coordinate (LDC)       at ($(cen)       + ( \WLD      , 0.0         )$) (LDC)   [\debugPoint];%
%	\coordinate (BCC)       at ($(cen)       + ( \WBC      , 0.0         )$) (BCC)   [\debugPoint];%
%	\coordinate (BDC)       at ($(cen)       + ( \WBD      , 0.0         )$) (BDC)   [\debugPoint];%
%	
%	\coordinate (LAD)       at ($(LAC)       + ( 0.0       ,-\picH*0.5   )$) (LAD)   [\debugPoint];%
%	\coordinate (LAU)       at ($(LAC)       + ( 0.0       , \picH*0.5   )$) (LAU)   [\debugPoint];%
%	\coordinate (LBDA)      at ($(LAC)       + ( 0.0       ,-\picH*0.4325)$) (LBDA)  [\debugPoint];%
%	\coordinate (LBD)       at ($(LBC)       + ( 0.0       ,-\picH*0.5   )$) (LBD)   [\debugPoint];%
%	\coordinate (LBDA)      at ($(LBC)       + ( 0.0       ,-\picH*0.4325)$) (LBDA)  [\debugPoint];%
%	\coordinate (LBU)       at ($(LBC)       + ( 0.0       , \picH*0.5   )$) (LBU)   [\debugPoint];%
%	\coordinate (LCD)       at ($(LDC)       + ( 0.0       ,-\picH*0.5   )$) (LCD)   [\debugPoint];%
%	\coordinate (LCDA)      at ($(LDC)       + ( 0.0       ,-\picH*0.4325)$) (LCDA)  [\debugPoint];%
%	\coordinate (LCU)       at ($(LDC)       + ( 0.0       , \picH*0.5   )$) (LCU)   [\debugPoint];%
	%coordinates, Box A Up, Tx_A
%	\coordinate (BAUCC)     at ($(BAC)       + ( 0.0       , \picH*0.3125)$) (BAUCC) [\debugPoint];%
%	\coordinate (BAUCL)     at ($(BAUCC)     + (-\BIPX     , 0.0         )$) (BAUCL) [\debugPoint];%
%	\coordinate (BAUCR)     at ($(BAUCC)     + ( \BIPX     , 0.0         )$) (BAUCR) [\debugPoint];%
%	\coordinate (BAUAC)     at ($(BAUCC)     + ( 0.0       ,-\BOH *0.4   )$) (BAUAC) [\debugPoint];%
%	\coordinate (BAUAL)     at ($(BAUAC)     + (-\BIPX     , 0.0         )$) (BAUAL) [\debugPoint];%
%	\coordinate (BAUAR)     at ($(BAUAC)     + ( \BIPX     , 0.0         )$) (BAUAR) [\debugPoint];%
%	\coordinate (BAUBC)     at ($(BAUCC)     + ( 0.0       ,-\BOH *0.2   )$) (BAUBC) [\debugPoint];%
%	\coordinate (BAUBL)     at ($(BAUBC)     + (-\BIPX     , 0.0         )$) (BAUBL) [\debugPoint];%
%	\coordinate (BAUBR)     at ($(BAUBC)     + ( \BIPX     , 0.0         )$) (BAUBR) [\debugPoint];%
%	\coordinate (BAUDC)     at ($(BAUCC)     + ( 0.0       , \BOH *0.2   )$) (BAUDC) [\debugPoint];%
%	\coordinate (BAUDL)     at ($(BAUDC)     + (-\BIPX     , 0.0         )$) (BAUDL) [\debugPoint];%
%	\coordinate (BAUDR)     at ($(BAUDC)     + ( \BIPX     , 0.0         )$) (BAUDR) [\debugPoint];%
%	\coordinate (BAUEC)     at ($(BAUCC)     + ( 0.0       , \BOH *0.4   )$) (BAUEC) [\debugPoint];%	
	
	%clipping
%	\clip ($(PLD) + ( 0.0 ,-0.5)$) rectangle ($(PRU) + ( 0.0 , 0.1)$);
	%basic picture
	
	% Slide numbering helper
	
%	\node[vis on=<1-1>,] () at ($(cen) + ( 0.00, 0.00)$) {\fontset{}\large\textbf{1}};
%	\node[vis on=<2-2>,] () at ($(cen) + ( 0.00, 0.00)$) {\fontset{}\large\textbf{2}};	
%	\node[vis on=<3-3>,] () at ($(cen) + ( 0.00, 0.00)$) {\fontset{}\large\textbf{3}};
%	\node[vis on=<4-4>,] () at ($(cen) + ( 0.00, 0.00)$) {\fontset{}\large\textbf{4}};
%	\node[vis on=<5-5>,] () at ($(cen) + ( 0.00, 0.00)$) {\fontset{}\large\textbf{5}};
%	\node[vis on=<6-6>,] () at ($(cen) + ( 0.00, 0.00)$) {\fontset{}\large\textbf{6}};
%	\node[vis on=<7-7>,] () at ($(cen) + ( 0.00, 0.00)$) {\fontset{}\large\textbf{7}};
%	\node[vis on=<8-8>,] () at ($(cen) + ( 0.00, 0.00)$) {\fontset{}\large\textbf{8}};
							
	%Vertical Lines============================================================%
	
	%Nodes=====================================================================%
	%node A Up, Tx_A-----------------------------------------------------------%
%	\node[vis on =<1->, nodeOuter]   (NBAU)   at (BAUCC)                      {};
%	\node[vis on =<1->, nodeInnerA]  (NBAUEC) at (BAUEC)                      {\fontset{}$\mathsf{Tx_A}$};
%	\node[vis on =<1->, nodeInnerBAC](NBAUDL) at (BAUDL)                      {\fontset{}\textbf{Input}};
%	\node[vis on =<1->, nodeInnerBAC](NBAUDR) at (BAUDR)                      {\fontset{}\textbf{Output}};
%	\node[vis on =<1->, nodeInnerBD] (NBAUBL) at (BAUBL)                      {\fontset{}$\mathsf{Inp_1}$};
%	\node[vis on =<1->, nodeInnerBA] (NBAUCR) at (BAUCR)                      {\color{black!50}\fontset{}$\mathit{Out_1}$};
%	\node[vis on =<1->, nodeInnerBA] (NBAUBR) at (BAUBR)                      {\fontset{}$\mathsf{Out_2}$};
%	\node[vis on =<1->, nodeInnerBA] (NBAUAR) at (BAUAR)                      {\fontset{}$\mathit{Chng}$};
	
	%Arrows depending on nodes=================================================%
%	\draw[lineIO , ->, vis on=<1-> ] (INPU)        .. controls ($(INPU)        + ( 0.00, 0.00)$) and ($(NBAUBL.west) + ( 0.00, 0.00)$) .. (NBAUBL.west); %str --> TxA
	
	%Coin Symbols==============================================================%
	%TxA-----------------------------------------------------------------------%
%	\pic[vis on=<1-> ,] () at ($(NBAUBL.east) + (-\BOH*0.125,-\BOH*0.0375)$) {coin={0.7pt}{0.05}{\BOH*0.12}{\colorCoin} {black!80!white}{black!90!white}};

	
	%Counter Node==============================================================%
%	\pic[vis on=<2->  ] ()        at ($(BCount) + ( \BOH*0.225/2, 0.05)$) {coin={1.0pt}{0.07}{\BOH*0.225}{\colorCoinY}{black!90!white}{black!100!white}};
\end{tikzpicture}
%}%
	\centering
	\caption{Overview of preprocessing of coinbase transactions. Although coinbase transactions have no inputs, we regard fees collected in a coinbase transactions's block as pseudo-inputs.}
\end{figure}

% \ifdefined\varInputAlgos%
\renewcommand{\arraystretch}{1.5}%
\begin{algorithm*}[!h]%
	\DontPrintSemicolon
	\caption{Whole-bill-approach: Measurement of $\protect\MCircWb$ for period $\perd$ with look-back window $\wndw$.}\label{algo:code_mcirc_wb}%
	\KwData{$\wndw_\Start$                                    	    \tcc*{Beginn of look-back window $w$}}%
	\KwData{$\TxW$                                              	\tcc*{Set of transactions in period $\wndw$}}%
%	\KwResult{$\MCircWb$                                           \tcc*{Money supply to be estimated within period $\wndw$}}%
%
	$ \MCircWb \gets {0} $											\tcc*{Money supply to be estimated within period $\wndw$}%
%
	\ForEach {$ t \in \TxW $                                      	\tcc*[]{Loop through transactions $t$ of period $\wndw$}}%
	{%
		\ForEach {$ i \in \InpT $                                 	\tcc*[]{Loop through inputs $i$ of transaction $t$}}%
		{%
			\If{%
				$\dateGen(i)<\wndw_\Start$\ or\ $\genByCoinbase(i)$\label{algo:code_mcirc_wbCond}\tcc*{Was $i$ generated before $\wndw_\Start$/from a coinbase transaction?}%
			}{%
				$\MCircWb \gets{\MCircWb + \valI(i)}$				\tcc*{Update money supply}%
			}%
		}%
	}%
	return $\MCircWb$                                             	\tcc*{Result: Return money in circulation within $\wndw$ as estimated for period \(\perd\) for \ac*{wba}}%
\end{algorithm*}%
\else%
\fi%
%% We don't need that anymore. Just keep it to look it up some time.
%
% line numbering
%
\renewcommand{\LinesNumbered}{%
	\setboolean{algocf@linesnumbered}{true}%
	\renewcommand{\algocf@linesnumbered}{\everypar={\nl}}}%
%
\let\oldnl\nl% Store \nl in \oldnl
\renewcommand{\nonl}{\renewcommand{\nl}{\let\nl\oldnl}}% Remove line number for one line
%
\ifdefined\varInputAlgos%
\renewcommand{\arraystretch}{1.90}%
%\todo{Do we mean non-recycling or not circulating?}%
\begin{algorithm*}[!h]%
	\DontPrintSemicolon
	\caption{Aggregating $\protect\MCirc$ for arbitrary type for given window $\wndw$ from genesis day to $\daytt_\maxtt$.%
	}\label{algo:code_aggr}%
%  \caption{Moved-coin-approach: Measurement $\protect\MCircM$ for $\mathtt{type}$ (\acs*{fifo}, \acs*{lifo}) within period $\perd$, window $\wndw$.%
%  }\label{algo:code_mcirc_mc}%
%
	\KwData{$ \MSetCbs$                                      \tcc*{Set of cumulated coinbase coins per day}}%
	\KwData{$ \MSetCirc$                                     \tcc*{Set of $\MCirc$ calculated per day, with $1$ day as lookback window}}%
	\KwData{$ \MSetCircAddA$                                 \tcc*{Set of $\MCirc$ calculated per day, with $\wndw$ day as lookback window}}%
	\KwData{$ \MSetCircAddB$                                 \tcc*{Set of $\MCirc$ output adjustments with $\wndw$ day as lookahead window}}%
	$\MSetCircAggr \gets \emptyset$                          \tcc*{Initialize Set of $\MCircAggr$}
%
	\ForEach{$d \in \{0,...,\daytt_\maxtt\}$%
	}{%
		$\MCircAggr \gets {0}$                               \tcc*{Initialize $\MCircAggr$ to be computed for this day}%
		                                                     \tcc*{Step 1: Use $\MCircAggr$ calculated for $d-1$ as base for aggregation at $d$, except for $d=0$}
		\If{$d > 0$}{%
			$\MCircAggr \gets \MSetCircAggr[d-1]$%
		}%
		                                                     \tcc*{Step 2: Add values of day $d$ to aggregation for $w$}%
		\eIf{$d < w$}{%
			$\MCircAggr \gets \MCircAggr + \MSetCbs[d]$      \tcc*{For $d<w$, only coinbase coins represent $\MCirc$}%
		}{%
			$\MCircAggr \gets \MCircAggr + \MSetCircAddA[d]$ \tcc*{For $d\ge{}w$, add $\MCirc$ of day $d$ calculated for lookback window $w$}%
		}%
	                                                         \tcc*{Step 3: Substract/adjust values of day $d-w$}%
		\If{$d \ge w$}{%
			$d_\mathtt{sub} = d - w$                         \tcc*{Get the first day of the last aggregate}%
			$\MCircAggr \gets \MCircAggr - \MSetCirc[d_\mathtt{sub}]$      \tcc*{Substract all }%
			$\MCircAggr \gets \MCircAggr + \MSetCircAddB[d_\mathtt{sub}]$  \tcc*{Add adjusted values}%
		}%
		$\MSetCircAggr \gets \MSetCircAggr \cup \MCircAggr$\\
	}%
	return $\MSetCircAggr$                                   \tcc*{Result: Return daily aggregated $\MCirc$ for window $w$}%
\end{algorithm*}%
\else%
\fi%
%GetMCirc
\ifdefined\varInputAlgos%
\renewcommand{\arraystretch}{1.5}%
\begin{algorithm*}[!h]%
	\DontPrintSemicolon
	\caption{$AdjustMCirc$: Adjustment value for $\MCirc$ for transaction $t$.}\label{algo:code_outs_spent_within}%
%	\KwData{$ \Out^{\cbs}$                                                         \tcc*{Set of maps of coinbase outputs per blockheight, that represent transaction fees}}%
%	\KwData{$ \TxD$                                                                \tcc*{Set of all transaction of same day to with $t$ belongs as well}}%
%	\KwData{$ \TxChecked$                                                          \tcc*{Helper to recall all spending transactions linked to same spent tx with already adjusted $\MCirc$}}%
	\KwData{$\TxIndexAdjustment$                                                    \tcc*{Set holding indexes of spending-transactions that have already been entered adjustment values.}}%
	$\MCircAdj \gets {0}$                                                          \tcc*{The adjustment value to be applied to $\MCirc$ after shifting $\wndw$}%
	\ForEach{$o \in \Out_{t}$}{%
		\lIf{$\isNotSpent(o)$}{continue}%
		$t_{\spendingtt} \gets {\spendingTx(o)}$                                   \tcc*{Get the transaction that spent $o$}%
		\lIf(                                                                      \tcc*[f]{If $t_\spendingtt$ entered an correction-value.}%
		){$\indextt(t_{\spendingtt})\,\in\,\TxChecked$%
		}{continue}%
		$\TxIndexAdjustment \gets {\TxIndexAdjustment\,\cup\,\{\indextt(t_{\spendingtt})\}}$       \tcc*[f]{Mark $t_\spendingtt$ as already adjusted by some output of $t$}\\%
%		\ForEach(                                                                  \tcc*[f]{Check if $t_\spendingtt$ was already adjusted previously by some other transaction of $\TxD$}%
%		){$i \in \Inp_{t_{\spendingtt}}$%
%		}{%
%			$t_{\spenttt} \gets {\spentTx(i)}$                                     \tcc*{Get the transaction that being spent by $i$}%
%			\If(                                                                   %\tcc*[f]{If $t_\spendingtt$ was already adjusted by some output of $t$}%
%			){$t_{\spenttt}\,\in\,\TxD\ \andbf\ \indextt(t_{\spenttt})\,<\,\indextt(t)$%
%			}{continue}%
%		}
%		$\MCircOld \gets{ GetMCirc\bigl(t,\,\typett,\,\wndw,\,\Out^{\cbs},\,\Out^{\selfchurn}\bigr)}$\\%
%		$\MCircNew \gets{ GetMCirc\bigl(t,\,\typett,\,\wndw+1,\,\Out^{\cbs},\,\Out^{\selfchurn}\bigr)}$\\%
		$\MCircOld \gets{ GetMCirc\bigl(t,\,\typett,\,\wndw\bigr)}$\\%
		$\MCircNew \gets{ GetMCirc\bigl(t,\,\typett,\,\wndw+1\bigr)}$\\%
		$\MCircAdj \gets {\MCircAdj\,+\,\bigl(\MCircNew\,-\,\MCircOld\bigr)}$      \tcc*{Update $\MCircAdj$}%
	}%
%
	return $\MCircAdj,~\TxIndexAdjustment$                                                             \tcc*{Result: The corrective value to be applied to  $\MCirc$ for a certain day and time window.}%
\end{algorithm*}%
\else%
\fi%
%
%AdjustMCirc
%%
% line numbering
%
\renewcommand{\LinesNumbered}{%
	\setboolean{algocf@linesnumbered}{true}%
	\renewcommand{\algocf@linesnumbered}{\everypar={\nl}}}%
%
\let\oldnl\nl% Store \nl in \oldnl
\renewcommand{\nonl}{\renewcommand{\nl}{\let\nl\oldnl}}% Remove line number for one line
%
\ifdefined\varInputAlgos%
\renewcommand{\arraystretch}{1.90}%
%\todo{Do we mean non-recycling or not circulating?}%
\begin{algorithm*}[!h]%
	\DontPrintSemicolon
	\caption{Aggregating $\protect\MCirc$ for arbitrary $\typett$ for given window length $\wndwLength$ from genesis day to $\daytt_\maxtt$.%
	}\label{algo:code_aggr}%
	%  \caption{Moved-coin-approach: Measurement $\protect\MCircM$ for $\mathtt{type}$ (\acs*{fifo}, \acs*{lifo}) within period $\perd$, window $\wndw$.%
	%  }\label{algo:code_mcirc_mc}%
	%
	
	\KwData{$\wndw$                                                        \tcc*{Lookback window $\wndw$}}%
	$ \MSetCbs \gets \emptyset$                                            \tcc*{Set of cumulated coinbase coins per day}%
	$ \MSetCirc_{1} \gets \emptyset$                                       \tcc*{Set of $\MCirc$ calculated per day, with $1$ day as lookback window}%
	$ \MSetCircLB \gets \emptyset$                                         \tcc*{Set of $\MCirc$ calculated per day, with $\wndw$ day as lookback window}%
	$ \MSetCircLA \gets \emptyset$                                         \tcc*{Set of $\MCirc$ output adjustments with $\wndw$ day as lookahead window}%
	$ \MSetCircAggr \gets \emptyset$                                       \tcc*{Initialize Set of $\MCircAggr$}
	$ \Out^{\cbs} \gets GetCoinbaseFees\bigl(\bigr)$                        \tcc*{Set of maps of coinbase outputs per blockheight, that represent transaction fees}%
	
	\ForEach{$d \in \{0,...,\daytt_\maxtt\}$%
	}{%

                
		$\MSetCirc_{1} \gets \MSetCirc_{1}\,\cup\,\{0\}$\\%
		$\MSetCircLB \gets \MSetCircLB\,\cup\,\{0\}$\\%
		$\MSetCircLA \gets \MSetCircLA\,\cup\,\{0\}$\\%
        	\tcc*{Windows for addition  of iteration-head, subtraction of iteration-tail \\ and adjustments (compare Figure \ref{fig:aggr}).}%
	        $\wndw^{\mathtt{lookback}=1} \gets d$\\%
          	$\wndw^{\mathtt{lookback}=\wndwLength} \gets [d_{\mathtt{end}}-\wndwLength,d_{\mathtt{end}}]$\\%
                $\wndw^{\mathtt{ahead}}   \gets  [d_{\mathtt{end}},d_{\mathtt{end}}+\wndwLength] $\\%
                % 
		\tcc*{Compute basic $\MCirc$ values for each transaction of day $d$}%
		\ForEach{$\TxD$}{%
			$\MSetCirc_{1}[d] \gets \MSetCirc_{1}[d]\,+\,GetMCirc\bigl(t,\,\wndw^{\mathtt{lookback=1}}\bigr)$\\%
			$\MSetCircLB[d] \gets \MSetCircLB[d]\,+\,GetMCirc\bigl(t,\,\wndw^{\mathtt{lookback=\wndwLength}}\bigr)$\\%
			$\MSetCircLA[d] \gets \MSetCircLA[d]\,+\,AdjustMCirc\bigl(t,\,\wndw^{\mathtt{ahead=\wndwLength}} \bigr)$\\%
		}

          
          }
	\ForEach{$\iterhead \in \{0,...,\daytt_\maxtt\}$%
	}{%
		$\MSetCircAggr \gets {\MSetCircAggr\,\cup\,\{0\}}$                 \tcc*{Initialize $\MCircAggr$ to be computed for this day}%
		\tcc*{Use $\MCircAggr$ calculated for $\iterhead-1$ as base for aggregation at $\iterhead$, except for $\iterhead=0$}
		\If{$\iterhead > 0$}{%
			$\MSetCircAggr[\iterhead] \gets \MSetCircAggr[\iterhead-1]$%
		}%
		\tcc*{Iteration step 1: Add values of day $\iterhead$ to aggregation for $w$}%
		\eIf{$\iterhead \ge w$}{%
			$\MSetCircAggr[\iterhead] \gets \MSetCircAggr[\iterhead]\,+\,\MSetCircLB[\iterhead]$     \tcc*{For $\iterhead\ge{}w$, add $\MCirc$ of day $\iterhead$ calculated for lookback window $w$}%
		}{%
			$\MSetCircAggr[\iterhead] \gets \MSetCircAggr[\iterhead]\,+\,\MSetCbs[\iterhead]$        \tcc*{For $\iterhead<w$, only coinbase coins represent $\MCirc$}%
		}%
		\tcc*{Iteration step 2: Substract/adjust values of day $\iterhead-w$}%
		\If{$\iterhead \ge w$}{%
			$\itertail\,=\,\iterhead\,\notin\,w$                                       \tcc*{Get the first day of the last aggregate}%
			$\MSetCircAggr[\iterhead] \gets \MSetCircAggr[\iterhead]\,-\,\MSetCirc_{1}[\itertail]$ \tcc*{Substract full $\MCirc$ value of day $\itertail$ and ... }%
			$\MSetCircAggr[\iterhead] \gets \MSetCircAggr[\iterhead]\,+\,\MSetCircLA[\itertail]$   \tcc*{... re-add adjusted values}%
		}%
	}%
	return $\MSetCircAggr$                                                 \tcc*{Result: Return daily aggregated $\MCirc$ for window $w$}%
\end{algorithm*}%
\else%
\fi%
%Aggregation

%%% Local Variables:
%%% mode: latex
%%% TeX-master: "../main.tex"
%%% End:


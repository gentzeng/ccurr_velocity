

%Segregation of money

% \section{Novel measures of velocity of money in effective circulation}
% \label{sec:cc_money_seg}%

% Based on the concept of velocity of money in effective circulation as
% clarified last section, we now propose algorithms to calculate velocity for
% UTXO-based cryptocurrencies. %

% \subsection{Determining money in circulation}
% \label{sec:cc_money_seg:sub:mcirc_concept}% 

% Conceptually, a monetary unit has circulated in a period if and only if it
% has been used as a medium of exchange.  Therefore, first a time window must
% be specified with respect to which monetary units are to be classified as
% \emph{circulating} or not.  %
% Money is referred to as circulating if it has been moved economically within
% the last day, month, year or generally any time period $\wndw$ covering
% $[\wndw_\Start , \wndw_\End]$. %

% To identify the fraction of the money supply which circulated within $\wndw$,
% we step trough every transaction recorded in period $\wndw$.  %
% Transactions spending outputs generated before $\wndw$ as well as outputs
% from coinbase transactions (seignorage) have the interpretation of bringing
% an amount into circulation that corresponds to the value of spent outputs.  %
% %%% NOTE: Why would fees be excluded here?? -> DISCUSS %%%hwe
% All inputs referring to UTXOs generated \emph{within} period $\wndw$, on the
% other hand, re-spend money which has already been counted as circulating.%
% \footnote{To develop an intuition, remember the example of sheep Eve being
%   sold and re-bought by Bob in $2020$.  %
%   This time Bitcoin was used and Alice held unspent outputs of $2$ bitcoin
%   from transactions $5$ years ago.  %
%   The transaction formed by Alice to buy sheep Eve, $Tx_1$, spends her
%   five-year-old output and generates a new one in 2020 which is controlled by
%   Bob.  %
%   When Bob buys Eve back, he pays with the output of $1$ bitcoin from $Tx_1$
%   in a second transaction ($Tx_2$).  %
%   For $\wndw = \perd = 2020$, the value of $Tx_1$ (1 bitcoin) originated
%   prior to 2020 and is thus money brought into circulation.  %
%   $Tx_2$, however, spent an output generated in 2020---an already counted
%   monetary unit.  %
% %In the example, thus $\MCirc_{2019}=1 \ttext{ Bitcoin}$. 
% } %
% % This can be illustrated with \reffig{mcirc_concept} where values are %
% % symbolized with coins. %
% % Here, we would need to determine, how many monetary units have made the %
% % transaction volume $8$ (sum of outputs in C, D, E and F excluding
% % self-churn) during period $\wndw$ possible. %
% % In this example, we would need to focus on transactions A, B and D in
% % contrast to transactions C and E which reused monetary unspent transaction
% % outputs that have been generated within period $\wndw$. %

% Note that we define the time window $\wndw$ within which spent coins are
% considered as \emph{circulating} distinct from the period $\perd$ for which
% the velocity measure is calculated, $ [\perd_\Start, \perd_{\End}]$.  This
% distinction can be parametrized via a maximum length $\wndwLength$ of the
% look-back window $\wndw$, where $\wndw_\Start = \perd_\Start - \wndwLength$
% and $\wndw_\End = \perd_\End$.  %

% The approach as characterized so far, however, does not account for two
% technical properties of UTXO-based cryptocurrencies: %
% First, transactions always spend prior transaction outputs in full.  %
% Second, there exists no attribution of individual outputs to the inputs of a
% transaction; thus it remains undefined which input corresponds to which
% output(s).  %

% The first property raises the question how to deal with transactions which
% send back change: %
% Should the sum of all inputs be considered \textit{in circulation,} as
% technically all was transferred---or should only the fraction sent to third
% parties be considered?  %
% We analyze both choices, naming the first \ac{wba} and the second
% \ac{mca}.  %
% They are visualized in \reffig{mcirc_concept}.  %
% The moved-coin approach considers only output $\mathsf{Out_1}$ of transaction
% $\mathsf{Tx_C}$ as circulating, not the change output.  %
% This approach captures the net economic value transferred to a third
% party.  %
% The \ac{wba} classifies the whole input of transaction $\mathsf{Tx_C}$ as
% circulating.  %
% This approach captures the amount of money that has been moved technically;
% it can also be interpreted as revealed to be available for transactions.  %

% In the \ac{mca}, the sum of inputs is counted as circulating only net of
% change outputs.  %
% However, due to the second property of UTXO blockchains ambiguous
% constellations can occur: If for a given transaction one input was generated
% within and one before $\wndw$, it remains unspecified which one corresponds
% to the change output.  %

% Transaction $\mathsf{Tx_F}$ in \reffig{mcirc_concept} illustrates the
% point.  %
% It has two inputs: $\mathsf{Inp_1}$ originated before period $\wndw$ and
% $\mathsf{Inp_2}$ originated within $\wndw$.  %
% If only amounts sent to third parties matter, it remains unclear which of the
% two inputs funded the change output.  %
% For $\mathsf{Inp_1}$, the transaction would not increase money in
% circulation.  %
% In contrast, if $\mathsf{Inp_2}$ funded the change output, the transaction
% would increase the amount in circulation by $\mathsf{Inp_1}$. %

% To resolve the ambiguity, an assignment rule between transaction inputs and
% outputs is required.  %
% % Utilizing the terminology of cost accounting,
% We consider both endpoints on the spectrum of the age of the input assigned
% to the change transaction and thus differentiate between \ac{lifo}, where
% oldest inputs get assigned to outputs first, and \ac{fifo}, where it is the
% other way around.  %

% Naturally, with the \ac{wba} this differentiation is void.  %
% Hence, we have three definitions of money in circulation, each a function of
% the activity window length $\wndwLength$: %
% Money in circulation for period $\wndw$ adopting the \ac{wba}
% ($ \MCircWbPWl $), and both the \ac{mca} with the \ac{lifo} rule
% ($ \MCircMlPWl $) and the \ac{fifo} rule ($ \MCircMfPWl $).

% \subsection{Definition of velocity measures}
% \label{sec:cc_money_seg:sub:}%

% Based on the above definitions of money in circulation, three variations of
% velocity can be calculated in accordance with \refequ{vcirc_concept}---one
% per money aggregate.  %
% %
% All measures capture the average number of peer-to-peer coin turnovers of
% effectively circulating monetary units in period $\perd$.  They concur in
% capturing on-chain liquidity; they differ \wrt to definitions of
% \textit{circulating monetary units} and assignment rules linking transaction
% inputs to outputs.  %
% % 
% The first measure is based on $\MCircWbP$ and simply calculated as %
% \begin{align}%
%   \label{equ:VCircWbEstPWl}
%   \VCircWbEstPWl = \frac{\langle\Pp,\Tp\rangle}{\MCircWbEstPWl}.%
% \end{align}%
% $\VCircWbP$ is recommended if a conservative measurement of coin turnover is
% sought or additional assumptions about how to link inputs to outputs should
% be avoided.
% % 
% The second and third measures are based on $\MCircMfPWl$ and $\MCircMlPWl$,
% respectively:  %
% \begin{align}%
%   \label{equ:VCircMfEstPWl}
%   \VCircMfEstPWl = \frac{\langle\Pp,\Tp\rangle}{\MCircMfEstPWl}, \\
%   \VCircMlEstPWl = \frac{\langle\Pp,\Tp\rangle}{\MCircMlEstPWl}.%
%   \label{equ:VCircMlEstPWl}
% \end{align}%
% %
% $\VCircMfEstP$ and $\VCircMlEstP$ are more stringent on the definition of
% money in circulation.  %
% Their monetary aggregates do not count ``touched'' funds, but only amounts
% transferred to somebody other than the sender.
% % Another advantage of the two measures is lower boundedness at 1 if \(\perd\) is set equal to \(\wndw\) which allows for
% % an intutive understanding of the circulation of coins.%
% % \footnote{Every transaction, by construction, can draw at most its own
% %   transaction value into circulation. %
% %   Transaction volume in both, numerator and denominator, is calculated free of change money.} % 
% However, this comes at the cost of the additional assumption with respect to
% the assignment rule between transaction inputs and outputs.  %



\subsection{Algorithmic implementation}
\label{sec:cc_money_seg:sub:mcirc_pract}%

\textbf{[TEMP] Notation:}:
\begin{itemize}
\item produce daily data, that is subscript: \(\perd\)
\item look back window lenght \(\wndwLength\)
\item lock back window \(\wndw\)
\item 
\end{itemize}

As illustrated in the conceptional description of the three velocity measures,
determining whether an unspent output was spent in a given time window is essential for
the chosen approach. %
While this is computationally feasible for small time windows, already for look-back windows
of a few days, naively looping through all transactions is problematic. %
For daily timeseries data and multiday look-back time windows the data for computations overlaps. %
The algorithm proposed in \cite{pernice2019cryptocurrencies} would thus loop over
and over again over the same transactions. %
\footnote{The published code on \url{https://github.com/wiberlin/ccurr_velocity} for a look-back time window of a week takes XYZ hours to terminate even if restricted on the first 3 years of the Bitcoin blockchain and run using all XY cores of a server with XY GB ram.}  

We thus propose an improved algorithmic implementation proceeding inductively. %
Our approach allows for reusing and adjusting information of calculations for prior time-windows instead of recalculating them entirely for every day \(\perd\). %

While the novel, inductive implementation allows effectively for larger time-windows \(\wndwLength\), we reuse the general approach for measuring money in circulation proposed in \cite{pernice2019cryptocurrencies}. %
We do adapt it, however, by adjusting and recombining its elements. %
Most importantly, we add a helper-construct measuring money that might be interpreted as monetary units that have been ``activated'' by the transactions of the certain day \( \perd \) that enters the time window after shifting it. %
The above helper-construct can be seen as the centerpiece to realize the inductive framework.
%The induction steps are summarized in \ref{algo:code_mcirc} and hold for all the three measures. %
Roughly spoken, a single inductive step takes the measure for money in circulation \( \MSetCircAggr \) for time window \(\wndw\) and then: %
\begin{enumerate}
\item Shifts window \(\wndw\) so that it starts now at \(\wndw_{\Start} + 1\) and ends at \(\wndw_{\End} + 1\),
\item adds the value of monetary units which are ``activated'' by the transactions of the time period covered by the period \(\wndw_{\End}\) to \(\wndw_{\End} + 1\) and thus \(\perd\) and %
\item subtract monetary units that are no longer part of the new time window. %
\end{enumerate}
The induction steps require pre-cashing of the helper-constructs for all days for which money in circulation is to be estimated \(\perd\). %
We will thus introduce its calculation first, followed by a detailed explanation of the induction algorithm . %



% Money in circulation under the \ac{wba} is measured as in
% \refalgo{code_mcirc_wb}.  %
% For every period $\wndw$, we loop over all transactions $t\in\TxW$ and add
% their inputs to circulating money if they either reference outputs from
% coinbase transactions, denoted by $\genByCoinbase(i)$, or outputs with
% timestamps $\dateGen(i)$ before the first timestamp $\wndw_\Start$ of period
% $\wndw$.  %
% %
% %  \footnote{In \reffig{mcirc_concept}, the algorithm would go trough %
% %  transaction C (adding output $\mathsf{1}$ of transaction B, referenced by input $\mathsf{1}$ %
% %  of transaction C), would then go to transaction D (finding no input at all), %
% %  then to transaction E (skipping output $\mathsf{1}$ of transaction E but adding output $\mathsf{1}$ %
% %  of transaction D as referenced by input $\mathsf{2}$ of transaction E) and so forth. %
% % } %
% %
% %\ifdefined\varInputAlgos%
\renewcommand{\arraystretch}{1.5}%
\begin{algorithm*}[!h]%
	\DontPrintSemicolon
	\caption{$GetInputSpentCase$: Check whether an input $i$ was generated before window $\wndw$ or stems from a coinbase transaction.}\label{algo:code_mcirc_cond}%
	\KwData{$i$                                    	                \tcc*{Input to be checked}}%
	\KwData{$\wndw_\Start$                                    	    \tcc*{Begin of look-back window $w$}}%
	$ \inpSpentCase \gets {0} $										\tcc*{Intialize value representing spent case for input $i$}%4
	\eIf(\tcc*[f]{Check whether input $i$ stems from a coinbase transaction}){%
		$\genByCoinbase(i)$%
	}{%
		\lIf(\tcc*[f]{Check whether input $i$ was generated before $\wndw_\Start$}){%
			$\dateGen(i) < \wndw_\Start$
		}{%
			$ \inpSpentCase \gets {3} $
		}
		\lElse{%
			$ \inpSpentCase \gets {2} $
		}%
	}{%
		\lIf(\tcc*[f]{Check whether input $i$ was generated before $\wndw_\Start$}){%
			$\dateGen(i) < \wndw_\Start$
		}{%
			$ \inpSpentCase \gets {1} $
		}
		\lElse{%
			$ \inpSpentCase \gets {0} $
		}%
	}%
	return $\inpSpentCase$                                          \tcc*{Result: Return value representing spent case for input $i$}%
\end{algorithm*}%
\else%
\fi%
%

% Measuring money in circulation under the \ac{mca} is depicted in
% \refalgo{code_mcirc_mc}.  %
% %
% As in \refalgo{code_mcirc_wb}, for time window $\wndw$ we loop over all
% transactions $t\in\TxP$ and add inputs based on the same core condition
% (compare lines~\ref{algo:code_mcirc_mc}-\ref{algo:code_mcirc_mcCond} and
% \ref{algo:code_mcirc_wb}-\ref{algo:code_mcirc_wbCond}).  %
% This time, however, only those inputs are regarded for further counting,
% which add up to the amount sent to third parties. Therefore, the calculated
% amount of money in circulation per transaction can be less or equal to the
% amount sent to third parties, but never more.
% The order of inputs to consider is determined by the \ac{lifo} or \ac{fifo}
% principle.  %
% For every transaction $t$ in time window $\wndw$, the amount sent to third
% parties is determined net of self-churn as
% $\valO^\sendToOthers(t) = \sum_{o\in{}\Out'_{t}} \valO(o)$ %
% where $\Out'_{t}$ denotes non-self-churn outputs of transaction $t$.  %
% If all outputs are identified as self-churn, $\valO^\sendToOthers(t) = 0$ and
% the algorithm continues with the next transaction.  %
% %
% If $\valO^\sendToOthers(t) > 0$, the algorithm collects input values in a
% vector $\InpSortedT$; they are sorted in either ascending (\ac{lifo}) or
% descending (\ac{fifo}) order \wrt to the timestamp when the UTXOs were
% generated.  %
% Then, looping over inputs $i$, input values $\valI(i)$ are added to
% $\MCircMPWl(t)$ if they meet the core condition (compare line 17) introduced
% in line 6 of \refalgo{code_mcirc_wb}.  %
% \footnote{Summands $\MCircMPWl(t)$ can be interpreted as money drawn into
%   effective circulation by transaction $t$.} %
% However, one additional condition applies: If the last added input would
% increase the summand $\MCircMPWl(t)$ beyond the value of outputs sent to
% third parties $\valO^\sendToOthers(t)$, we only add up to the latter
% amount.  %
% % This effectively adds only the necessary fraction of the inputs of
% % transaction $t$, generated before window $w$ or from coinbase transactions,
% % that were required to match $\valO^\sendToOthers(t)$.  %

%   % The components $\MCircMPWl(t)$ are consecutively summing all transaction
%   % values to calculate money in circulation $\MCircMPWl$ for the given period
%   % $p$ with respect to time window $\wndw$ as well as the applied sorting type.

% % \par While our algorithms theoretically might be applied to any window length $\wndwLength$, in practice they lack performance for large windows. %
% % Additional methods of optimization might be needed, to apply the method windows larger then 30 days. %



% 
%%% Local Variables:
%%% mode: latex
%%% TeX-master: "../main"
%%% End:
%


%\ifdefined\varInputAlgos%
\renewcommand{\arraystretch}{1.5}%
\begin{algorithm*}[!h]%
	\DontPrintSemicolon
	\caption{$GetInputSpentCase$: Check whether an input $i$ was generated before window $\wndw$ or stems from a coinbase transaction.}\label{algo:code_mcirc_cond}%
	\KwData{$i$                                    	                \tcc*{Input to be checked}}%
	\KwData{$\wndw_\Start$                                    	    \tcc*{Begin of look-back window $w$}}%
	$ \inpSpentCase \gets {0} $										\tcc*{Intialize value representing spent case for input $i$}%4
	\eIf(\tcc*[f]{Check whether input $i$ stems from a coinbase transaction}){%
		$\genByCoinbase(i)$%
	}{%
		\lIf(\tcc*[f]{Check whether input $i$ was generated before $\wndw_\Start$}){%
			$\dateGen(i) < \wndw_\Start$
		}{%
			$ \inpSpentCase \gets {3} $
		}
		\lElse{%
			$ \inpSpentCase \gets {2} $
		}%
	}{%
		\lIf(\tcc*[f]{Check whether input $i$ was generated before $\wndw_\Start$}){%
			$\dateGen(i) < \wndw_\Start$
		}{%
			$ \inpSpentCase \gets {1} $
		}
		\lElse{%
			$ \inpSpentCase \gets {0} $
		}%
	}%
	return $\inpSpentCase$                                          \tcc*{Result: Return value representing spent case for input $i$}%
\end{algorithm*}%
\else%
\fi%
%%
%\ifdefined\varInputAlgos%
\renewcommand{\arraystretch}{1.5}%
\begin{algorithm*}[!h]%
	\DontPrintSemicolon
	\caption{$GetCoinbaseFees$: Precompute $\Out^{\cbs}$: outputs of coinbase transactions representing collected fees.}\label{algo:code_cbs_out}%
	$\Out^{\cbs} \gets{\emptyset}$                                      \tcc*{The set of maps of coinbase tx outputs representing collected fees per blockheight}%
	\ForEach{$\mathrm{blockheight}\ bh\ \mathrm{of\ a\ given\ Blockchain}$}{%
		$\Out^{\cbs} \gets{\Out^{\cbs}\,\cup\,\{\emptyset\}}$           \tcc*{Append an empty map for this blockheight}%
		$ \outSum  \gets{0}$                                            \tcc*{Sum of regarded outputs}%
		$ \feeRem   \gets{0}$                                           \tcc*{Value of last output (partly) representing collected fees}%
		$ \minedRem \gets{0}$                                           \tcc*{Value of first output (partly) representing mined coins}%
		$ \feeSum   \gets{\sum_{f\in{}\Fee(bh)} \val(f)}$               \tcc*{Sum of all fees collected in block}%
		$ \Out'_{t}\gets{sortSpentOnly(\OutT)}$                         \tcc*{Sort ascending by indices of transactions spending the outputs and discard utxos}%
		\lIf(\tcc*[f]{Continue if all outputs are unspent}){%
			$\Out'_{t} = \emptyset$%
		}{continue}%
		\ForEach{$o \in \Out'_{t}$}{%
			$\keyOut \gets {(\addr(o),\ \val(o))}$                       \tcc*{Map key for last appended map $\Out^{\cbs}[bh]$, supposed to be unique}%
			$\Out^{\cbs}[bh][\keyOut] \gets {\indextt(o)}$  \tcc*{Add map entry, where $\keyOut$ is mapped to index of $o$ within $\Out'_{t}$}%
			$\outSum \gets {\outSum\,+\,\val(o)}$                         \tcc*{Update $\outSum$}%
			\If(\tcc*[f]{If we exceded all outputs representing fees}%
			){$\outSum \ge \feeSum$}{%
				$\minedRem \gets {\outSum\,-\,\feeSum}$                   \tcc*{Compute value of first output (partly) representing mined coins}%
				$\feeRem   \gets {\val(o)\,-\,\minedRem}$                 \tcc*{Compute value of last output (partly) representing collected fee}%
				break
			}%
		}%
		$\Out^{\cbs}[bh][\mathtt{"minedRemIndex"}] \gets {\indextt(o)}$ \tcc*{Add index of last added output $o$ within $\Out'_{t}$}%
		$\Out^{\cbs}[bh][\mathtt{"minedRem"}] \gets {\minedRem}$        \tcc*{Add value of first output (partly) representing mined coins}%
		$\Out^{\cbs}[bh][\mathtt{"feeRem"}] \gets {\feeRem}$            \tcc*{Add value of last output (partly) representing collected fees}%
	}%
%
	return $\Out^{\cbs}$                                                \tcc*{Result: The set of maps of coinbase tx outputs representing collected fees}%
\end{algorithm*}%
\else%
\fi%
%%
%\ifdefined\varInputAlgos%
\renewcommand{\arraystretch}{1.5}%
\begin{algorithm*}[!h]%
	\DontPrintSemicolon
	\caption{$AdjustMCirc$: Adjustment value for $\MCirc$ for transaction $t$.}\label{algo:code_outs_spent_within}%
	$\MCircAdj \gets {0}$                                                          \tcc*{The adjustment value to be applied to $\MCirc$ after shifting $ \wndw $}%
	\ForEach{$o \in \Out_{t}$}{%
          \lIf{$\isNotSpent(o)$}{continue}%
          \lIf{$\mathtt{date}(\spendingTx(o)) \in \wndw^{\mathtt{ahead}=\wndwLength} $}{
            $\MCircAdj \gets {\MCircAdj\,+\,\val(o)}$      \tcc*{Update $\MCircAdj$}%
            }%
	}%
%
	return $\MCircAdj$                                                             \tcc*{Result: The corrective value to be applied to  $\MCirc$ for a certain day and time window.}%
\end{algorithm*}%
\else%
\fi%
%
%
%
% line numbering
%
\renewcommand{\LinesNumbered}{%
	\setboolean{algocf@linesnumbered}{true}%
	\renewcommand{\algocf@linesnumbered}{\everypar={\nl}}}%
%
\let\oldnl\nl% Store \nl in \oldnl
\renewcommand{\nonl}{\renewcommand{\nl}{\let\nl\oldnl}}% Remove line number for one line
%
\ifdefined\varInputAlgos%
\renewcommand{\arraystretch}{1.90}%
%\todo{Do we mean non-recycling or not circulating?}%
\begin{algorithm*}[!h]%
	\DontPrintSemicolon
	\caption{Aggregating $\protect\MCirc$ for arbitrary type for given window $\wndw$ from genesis day to $\daytt_\maxtt$.%
	}\label{algo:code_aggr}%
%  \caption{Moved-coin-approach: Measurement $\protect\MCircM$ for $\mathtt{type}$ (\acs*{fifo}, \acs*{lifo}) within period $\perd$, window $\wndw$.%
%  }\label{algo:code_mcirc_mc}%
%
	\KwData{$ \MSetCbs$                                      \tcc*{Set of cumulated coinbase coins per day}}%
	\KwData{$ \MSetCirc$                                     \tcc*{Set of $\MCirc$ calculated per day, with $1$ day as lookback window}}%
	\KwData{$ \MSetCircAddA$                                 \tcc*{Set of $\MCirc$ calculated per day, with $\wndw$ day as lookback window}}%
	\KwData{$ \MSetCircAddB$                                 \tcc*{Set of $\MCirc$ output adjustments with $\wndw$ day as lookahead window}}%
	$\MSetCircAggr \gets \emptyset$                          \tcc*{Initialize Set of $\MCircAggr$}
%
	\ForEach{$d \in \{0,...,\daytt_\maxtt\}$%
	}{%
		$\MCircAggr \gets {0}$                               \tcc*{Initialize $\MCircAggr$ to be computed for this day}%
		                                                     \tcc*{Step 1: Use $\MCircAggr$ calculated for $d-1$ as base for aggregation at $d$, except for $d=0$}
		\If{$d > 0$}{%
			$\MCircAggr \gets \MSetCircAggr[d-1]$%
		}%
		                                                     \tcc*{Step 2: Add values of day $d$ to aggregation for $w$}%
		\eIf{$d < w$}{%
			$\MCircAggr \gets \MCircAggr + \MSetCbs[d]$      \tcc*{For $d<w$, only coinbase coins represent $\MCirc$}%
		}{%
			$\MCircAggr \gets \MCircAggr + \MSetCircAddA[d]$ \tcc*{For $d\ge{}w$, add $\MCirc$ of day $d$ calculated for lookback window $w$}%
		}%
	                                                         \tcc*{Step 3: Substract/adjust values of day $d-w$}%
		\If{$d \ge w$}{%
			$d_\mathtt{sub} = d - w$                         \tcc*{Get the first day of the last aggregate}%
			$\MCircAggr \gets \MCircAggr - \MSetCirc[d_\mathtt{sub}]$      \tcc*{Substract all }%
			$\MCircAggr \gets \MCircAggr + \MSetCircAddB[d_\mathtt{sub}]$  \tcc*{Add adjusted values}%
		}%
		$\MSetCircAggr \gets \MSetCircAggr \cup \MCircAggr$\\
	}%
	return $\MSetCircAggr$                                   \tcc*{Result: Return daily aggregated $\MCirc$ for window $w$}%
\end{algorithm*}%
\else%
\fi%
%
%%
% line numbering
%
\renewcommand{\LinesNumbered}{%
	\setboolean{algocf@linesnumbered}{true}%
	\renewcommand{\algocf@linesnumbered}{\everypar={\nl}}}%
%
\let\oldnl\nl% Store \nl in \oldnl
\renewcommand{\nonl}{\renewcommand{\nl}{\let\nl\oldnl}}% Remove line number for one line
%
\ifdefined\varInputAlgos%
\renewcommand{\arraystretch}{1.90}%
%\todo{Do we mean non-recycling or not circulating?}%
\begin{algorithm*}[!h]%
	\DontPrintSemicolon
	\caption{Aggregating $\protect\MCirc$ for arbitrary $\typett$ for given window length $\wndwLength$ from genesis day to $\daytt_\maxtt$.%
	}\label{algo:code_aggr}%
	%  \caption{Moved-coin-approach: Measurement $\protect\MCircM$ for $\mathtt{type}$ (\acs*{fifo}, \acs*{lifo}) within period $\perd$, window $\wndw$.%
	%  }\label{algo:code_mcirc_mc}%
	%
	
	\KwData{$\wndw$                                                        \tcc*{Lookback window $\wndw$}}%
	$ \MSetCbs \gets \emptyset$                                            \tcc*{Set of cumulated coinbase coins per day}%
	$ \MSetCirc_{1} \gets \emptyset$                                       \tcc*{Set of $\MCirc$ calculated per day, with $1$ day as lookback window}%
	$ \MSetCircLB \gets \emptyset$                                         \tcc*{Set of $\MCirc$ calculated per day, with $\wndw$ day as lookback window}%
	$ \MSetCircLA \gets \emptyset$                                         \tcc*{Set of $\MCirc$ output adjustments with $\wndw$ day as lookahead window}%
	$ \MSetCircAggr \gets \emptyset$                                       \tcc*{Initialize Set of $\MCircAggr$}
	$ \Out^{\cbs} \gets GetCoinbaseFees\bigl(\bigr)$                        \tcc*{Set of maps of coinbase outputs per blockheight, that represent transaction fees}%
	
	\ForEach{$d \in \{0,...,\daytt_\maxtt\}$%
	}{%
          	$\\wndw^{\mathtt{lookback=1}} \gets [d_{\mathtt{end}}-1,d_{\mathtt{end}}]$       \tcc*{TODO COMMENT}%
          	$\wndw^{\mathtt{lookback=\wndwLength}} \gets [d_{\mathtt{end}}-\wndwLength,d_{\mathtt{end}}]$       \tcc*{TODO COMMENT}%
                $\wndw^{\mathtt{ahead}}   \gets  [d_{\mathtt{end}},d_{\mathtt{end}}+\wndwLength] $      \tcc*{The look-ahead window for adjustments}%
                %
		$\MSetCirc_{1} \gets \MSetCirc_{1}\,\cup\,\{0\}$\\%
		$\MSetCircLB \gets \MSetCircLB\,\cup\,\{0\}$\\%
		$\MSetCircLA \gets \MSetCircLA\,\cup\,\{0\}$\\%
                % 
		\tcc*{Compute basic $\MCirc$ values for each transaction of day $d$}%
		\ForEach{$\TxD$}{%
			$\MSetCirc_{1}[d] \gets \MSetCirc_{1}[d]\,+\,GetMCirc\bigl(t,\,\wndw^{\mathtt{lookback=1}}\bigr)$\\%
			$\MSetCircLB[d] \gets \MSetCircLB[d]\,+\,GetMCirc\bigl(t,\,\wndw^{\mathtt{lookback=\wndwLength}}\bigr)$\\%
			$\MSetCircLA[d] \gets \MSetCircLA[d]\,+\,AdjustMCirc\bigl(t,\,\wndw^{\mathtt{ahead=\wndwLength}} \bigr)$\\%
		}

          
          }
	\ForEach{$\iterhead \in \{0,...,\daytt_\maxtt\}$%
	}{%
		$\MSetCircAggr \gets {\MSetCircAggr\,\cup\,\{0\}}$                 \tcc*{Initialize $\MCircAggr$ to be computed for this day}%
		\tcc*{Use $\MCircAggr$ calculated for $\iterhead-1$ as base for aggregation at $\iterhead$, except for $\iterhead=0$}
		\If{$\iterhead > 0$}{%
			$\MSetCircAggr[\iterhead] \gets \MSetCircAggr[\iterhead-1]$%
		}%
		\tcc*{Iteration step 1: Add values of day $\iterhead$ to aggregation for $w$}%
		\eIf{$\iterhead \ge w$}{%
			$\MSetCircAggr[\iterhead] \gets \MSetCircAggr[\iterhead]\,+\,\MSetCircLB[\iterhead]$     \tcc*{For $\iterhead\ge{}w$, add $\MCirc$ of day $\iterhead$ calculated for lookback window $w$}%
		}{%
			$\MSetCircAggr[\iterhead] \gets \MSetCircAggr[\iterhead]\,+\,\MSetCbs[\iterhead]$        \tcc*{For $\iterhead<w$, only coinbase coins represent $\MCirc$}%
		}%
		\tcc*{Iteration step 2: Substract/adjust values of day $\iterhead-w$}%
		\If{$\iterhead \ge w$}{%
			$\itertail\,=\,\iterhead\,\notin\,w$                                       \tcc*{Get the first day of the last aggregate}%
			$\MSetCircAggr[\iterhead] \gets \MSetCircAggr[\iterhead]\,-\,\MSetCirc_{1}[\itertail]$ \tcc*{Substract full $\MCirc$ value of day $\itertail$ and ... }%
			$\MSetCircAggr[\iterhead] \gets \MSetCircAggr[\iterhead]\,+\,\MSetCircLA[\itertail]$   \tcc*{... re-add adjusted values}%
		}%
	}%
	return $\MSetCircAggr$                                                 \tcc*{Result: Return daily aggregated $\MCirc$ for window $w$}%
\end{algorithm*}%
\else%
\fi%
%

%%% Local Variables:
%%% mode: latex
%%% TeX-master: "../main.tex"
%%% End:


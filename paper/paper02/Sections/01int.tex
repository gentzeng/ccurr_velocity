% Introduction
\section{Introduction}\label{sec:intro}%
Velocity of money plays a key role in traditional monetary economics %
since having been popularized by \cite{fisher1911equation} over a century ago. %
\parRemain{%
  Broadly speaking, velocity of money denotes the average number of
  transactions per monetary unit within a certain time period.%
  \footnote{We refer to its definition arising from the ``transaction form''
    of the quantity theory of money.}  %
  In the quantity theory of money, velocity is related to the price level. %
  While empirical studies frequently apply this concept to cryptocurrencies,
  surprisingly few find a significant relationship between velocity and
  prices.  %
  We take this discrepancy as occasion to evaluate current approaches to
  quantify the velocity of money for cryptocurrencies, and propose a novel
  one. %
}%

Until recently, meaningful measures for velocity of cryptocurrencies did not
exist, and most studies resorted to proxy variables.%
\footnote{We use \emph{measure} and \emph{estimator} interchangeably but
  contrast them to the terms \emph{proxy variable} or \emph{approximating
    variable}.  The former quantify the addressed concept directly (\eg
  length with a yardstick), while the latter rely on the quantification of a
  distinct concept which is assumed to be correlated with the one sought (\eg
  wealth via horsepower of owned cars).} %
\parRemain{%
  Recent years saw first advances to measure---instead of approximate---the
  velocity of money. %
  In \cite{athey2016bitcoin} and \cite{bolt2016value} the
  quantity equation of money has first been considered
  to measure velocity as the ratio of transaction volume and money supply. %
  In \cite{athey2016bitcoin}, however, this approach is modified to create a measure handling
  the change transactions in cryptocurrency systems. %
  While \cite{athey2016bitcoin} and later \cite{kalodner2017blocksci} focused
  on adjusting the transaction volume in the above ratio, we complement their
  approach by adjusting the money supply. %
}%

Money in effective circulation should be differentiated from money held for
long-term investment or speculation. %
\parRemain{%
  Not only does the total monetary aggregate contain technically
  dysfunctional money (burnt coins), a major portion of cryptocurrency is
  stored unused over long time periods (compare \cite{glaser2014bitcoin} or
  \cite{kalodner2017blocksci}). %
  Economists like \cite{fisher1911equation}, \cite{commons2003institutional}
  or \cite{keynes1930treatise} have argued to exclude such funds and focus on
  money in circulation. %
  To our knowledge, \cite{bolt2016value} and \cite{athey2016bitcoin} were
  first to apply this distinction in theoretical cryptocurrency pricing
  models. %
  Both link feedback effects from speculation and price levels to a reduction
  of coins in effective circulation. %
  In \cite{bolt2016value} velocity of money is explicitly defined as based on
  the component of coin supply in effective circulation. %
  In this paper, we operationalize this definition for velocity
  measurement. %
}%

In implementing this concept, we make common implicit assumptions explicit. %
\parRemain{%
  For example, the separation of money into \emph{hoarded} or
  \emph{circulating} depends on the choice of a time window. %
  Tokens can be defined as circulating if moved within the last day, month,
  year or any other period. %
%
  The choice of \cite{athey2016bitcoin} and \cite{kalodner2017blocksci},
  defining money in circulation as the total coin supply, implies an infinite
  time window. %
  The other extreme might be a very restrictive definition requiring coins to
  be moved within the period for which velocity is measured. %
  As the optimal time-window might depend on the respective use case, we
  operationalize a velocity measure for UTXO-based%
  \footnote{Descendants of Bitcoin's approach to build a transaction graph
    are known as \textit{UTXO-based cryptocurrencies}.  See
    \refsec{concept_utxo} for details.} %
  cryptocurrencies as a function of the respective time-window. %
}%

Subsequently, we apply our approach to Bitcoin and compare a variety of
potential proxy variables to measures characterizing the two extremes of the
design space. %
\parRemain{%
  Measuring the goodness of fit from a variety of perspectives, we show that
  the most common proxy-variable, \ac{cdd}%
  \footnote{The variable is discussed in
    \refsec{results:sub:approx_crypto:subsub:bdd}.
  % \Acl{cdd} %
  % of a transaction roughly can be interpreted as the product of value %
  % of spent coins and the days since these coins have been used. %
  % The measure aggregates these products over the transactions %
  % of a certain period.%
  } %
  in the vast majority of tests shows higher approximation errors than the
  simple ratio of unadjusted, on-chain transaction volume and total coin
  supply as shown by a series of \ac{mcs} tests. %
  As the majority of research opted for \ac{cdd}, our results might suggest a
  reason for the unexpectedly missing relation between velocity and prices in
  most studies. %
}%

%use in ieee
%\newpage

Our implementation is based on the open-source blockchain parser
\emph{BlockSci}%
\footnote{\url{https://citp.github.io/BlockSci/index.html}}. %
\parRemain{%
  The codebase to calculate the evaluated velocity measures for UTXO-based
  cryptocurrencies is re-usable and will be openly available after
  publication. %
  In summary, we offer three contributions to research on cryptocurrencies:
\begin{itemize}\setlength{\itemsep}{0pt}
\item a review of approaches to quantify velocity,%
  \footnote{Refer to Appendix \ref{sec:summ-veloc-meas} for a condensed
    summary of all approaches.}
\item novel measures based on money in circulation, and%
%\item a reusable toolset to estimate velocity
\item an evaluation of common approximation methods.%
\end{itemize}
}%

% The remainder of this paper is structured as follows. In \refsec{lit}, we review related literature. %
% The theoretic foundations are introduced in \ref{concepts} and~\ref{sec:concept_utxo}. %
% Practical issues of the on-chain transaction volume are discussed in \refsec{particularities_txvol}. %
% Subsequently, we review the recently emerged estimation methods in \refsec{oldmeas}. %
% We introduce the concept of an estimator based on money in effective circulation in \refsec{newmeas} and propose the new operationalization for identifying the respective monetary aggregate in \refsec{cc_money_seg} and finally introduce three variations of the measure in \refsec{newest}. %
% \refsec{approx_crypto} introduces used and proposed approximation methods for the velocity of money, which are lastly compared to the implemented estimators in \refsec{results}. %
% We conclude with \refsec{concl}.


%%% Local Variables:
%%% mode: latex
%%% TeX-master: "../main"
%%% End:



%Fischer and Stuff
\section{Theoretical concepts of measures for the velocity of money}
\label{sec:concepts}

% To facilitate an informed discussion of measures for the velocity of money,
% we first clarify the use of technical terms for the theoretical concepts. %

At its core, velocity of money refers to the average number of turnovers per
monetary unit within a period of time. %
This definition stems from the \emph{transaction form} of the quantity theory
of money as formalized by \cite{fisher1911equation}.%
\footnote{
  % The concept of relating money flows to monetary flows of goods and services has %
  % been put to writing first by \cite{humes1752ofmoney} but might have even older %
  % roots~\cite{volckart1997early}. %
  % Popularized by~\cite{fisher1911equation}, many different models arose. %
  % These are are summarized under the term \textit{quantity equations of money} today. %
  The transaction form highlights the use of money as a medium of exchange; %
  other forms stress different characteristics.  All forms, however, relate %
  money flows to real transactions. %
  While the transaction form is an identity and thus by definition correct, %
  it is nowadays mostly considered impractical for use in monetary %
  policy-making \citep[cf.][]{friedman2017quantity}. %
  General criticism was spurred by \cite{keynes2018general, tobin1989money, hansen1957american}, who reasoned that an unstable, %
  endogenously forming velocity of money would render the equations useless %
  for steering inflation. %
  The relation of monetary aggregates and price levels is disputed until this day \citep[cf.][]{dwyer1988money, mccallum2001monetary, gali2002new, bachmeier2005predicting, favara2009reconsidering}. 
  % As of definitional challenges to some of the theories %
  % central variables, the theory might better be formulated using income
  % transactions \cite{angell1936behavior} or focusing on the usefulness of
%   money as an asset rather than the mechanical payment process \cite{tobin1958liquidity}. %
}
%
The central equation of the theory equates the flows of real transactions,
given by the scalar product $\langle\Pp,\Tp\rangle$ of prices $\Pp$ and
transaction volumes $\Tp$, to total money flows, equal to the product of the
money supply $\Mp$ and its velocity \(\Vp\), where \(\perd\) denotes the time
period considered. %
With $n\in\mathbb{N}$ this amounts to
%
\begin{align}%
\label{eq:fisher}
  \Mp \Vp=\langle\Pp,\Tp\rangle\ \text{with}\ \Mp, \Vp \in \mathbb{R},\ 
  \text{and}\ \Pp,\Tp \in \mathbb{R}^{n}.%
\end{align}
The scalar product on the right-hand side is referred to as the \textit{price
  sum}.  In this product, \( \Pp=( \Pp_{1},\Pp_{2}, \cdots , \Pp_{n} ) \)
denotes a vector of prices $\Pp_t$ of transacted goods and services in
transaction $t$ during period $\perd$.  Transaction volumes \( \Tp \) are
given in units of goods and services. %
They are conceptualized as the vector %
\( \Tp=( \Tp_{1},\Tp_{2}, \cdots , \Tp_{n} ) \) %
with volume $\Tp_t$ in transaction $t$ in period $\perd$.  On the left-hand
side, $\Mp$ stands for the number of all units of money supply available in
period $\perd$. %
$\Vp$ denotes the velocity of money.%
\footnote{\label{sheep-note} To see this equation less abstractly, imagine an
  economy with only 2 gold coins, Alice, Bob and sheep Eve. %
  Alice owns the 2 gold coins and Bob the sheep Eve. %
  In $2020$, Alice buys Eve from Bob for 1 gold coin. %
  Later in the year, Bob regrets his decision and buys Eve back for the same
  price---Alice receives her coin back. %
  The quantity equation states the following:\\%
  %
  $ \displaystyle
  \begin{aligned}%
    2\,\coins \, \cdot \, V_{\mathtt{2020}} = %
    \langle\begin{pmatrix}%
    	1\,\frac{\coins}{\sheep} \\%
    	1\,\frac{\coins}{\sheep}%
    \end{pmatrix}, % 
    \begin{pmatrix}%
    	1\,\sheep \\%
    	1\,\sheep%
    \end{pmatrix}\rangle %
    = 2\,\coins.%
  \end{aligned} %
  $ \\%
  Now the velocity $V_{\mathtt{2020}}$ of the economy's money can be backed out
  as $1$.%
} %
%
While \(\Tp\), \(\Vp\) and \(\Pp\) are measured over a time period, \(\Mp\) %
is a point-in-time measure. %
To simplify, we assume the money supply is fixed during period \(\perd\)
and record it at the period's beginning \(\perd_\Start\). %

To develop intuition for velocity \(\Vp\), it can be viewed as weighted
average number of turnovers of all monetary units
\citep[cf.][]{friedman2017quantity}. %
The weights are derived from sorting the units into groups $g\in{}\Gp$ with
respect to their number of turnovers $\vp_g$ during period $\perd$. %
Velocity $\Vp$ then is%
\begin{align}\label{eq:velo_concept}%
  \Vp = \sum_{g\in\Gp}% 
  \Bigl(%
  \vp_g \cdot \frac{\Np_g}{\sum_{g\in\Gp} \Np_g}%
  \Bigr)%
  ,%
\end{align}%
with $\Np_g$ monetary units in group $g$ in period $\perd$. %
Velocity thus is the sum of turnover numbers~$\vp_g$, weighted by their
respective fractions.%
\footnote{%
  Returning to the example in footnote~\ref{sheep-note}, one coin was turned
  over twice---and one not at all. %
  Thus %
  $V_{\mathtt{2020}}= 0 \cdot \frac{%
    1\,\coins%
  }{%
    2\,\coins%
  } + 2\,\cdot \frac{%
    1\,\coins%
  }{%
    2\,\coins%
  }$. %
  This shows how the quantity equation is an identity and an implicit
  definition of velocity (rather than a testable statement, compare
  \cite{friedman2017quantity}).%
} %
While this definition is intuitive, it cannot be used to measure velocity in
practice. %
Turnover numbers per monetary unit are neither recorded for fiat currencies,
nor can they be inferred unambiguously for UTXO-based cryptocurrencies
% by counting transfers of coins
(compare \refsec{concept_utxo}). %
In practice, the velocity of money is thus backed out of \refequ{fisher}:
$V_\perd = \langle\Pp,\Tp\rangle/\Mp$. %

For cryptocurrencies, this appears a simple task, using on-chain transaction
volume and total coin supply. %
However, due to their technical implementation, transaction volumes recorded
on-chain are distorted. %
The next section therefore discusses the relevant subtleties of UTXO-based
cryptocurrency systems. %


%%% Local Variables:
%%% mode: latex
%%% TeX-master: "../main"
%%% End:

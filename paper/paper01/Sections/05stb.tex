%State-of-the-art estimations
\section{Self-churn and Clustering}
\label{sec:particularities_txvol}%
% 
As discussed in the last section, by construction many cryptocurrency
transactions contain outputs sending a fraction of the value back to the
sender. %
Such change outputs as well as other self-churn ought to be excluded from
transaction volume: %
%
\blockquote[%
\cite{fisher1922purch}%
]{%
  What is desired is the rate at which money is used for purchasing %
  goods, not for making change
}. %

\subsection{Transaction volume inflated by self-churn}
\label{sec:particularities_txvol:inflated}% 
While the transaction volume in principle can be calculated accumulating the
output values $\valO(o)$%
\footnote{To simplify notation, we denote many variables as functions of
  transactions, outputs or inputs. %
  For example, function $\valO(o)$ extracts the output value of output \(o\)
  of transaction \(t\) in period \(\perd\).} %
of all transactions $t$ recorded within period $\perd$ this yields an
inflated aggregate $\langle\Pp',\Tp'\rangle$ when defined as
\begin{align}%
  \langle\Pp',\Tp'\rangle = \sum_{t\in\TxP}\sum_{o\in\Out_\mathtt{t}} \valO(o),%
\end{align} %
with $o\in\Out_\perdT$ the set of all outputs of transaction $t$ in period
$\perd$ and \(\TxP\) the set of all transactions $t$ recorded within %
period $\perd$.  Thus, this volume needs to be adjusted. %
Defining $\Out^{\selfchurn}$ as the set of all self-churn outputs, the
accumulated transaction volume from these outputs $\Op$ can be obtained by
summing the individual self-churn outputs $c \in \Out^{\selfchurn}$ as%
\begin{align}%
  \Op = \sum_{t\in\TxP} \sum_{o\in\Out^{\selfchurn}} \valO(o).%
\end{align}%
Hence a corrected transaction volume can be calculated as %
$\langle\Pp,\Tp\rangle= \langle\Pp',\Tp'\rangle - \Op$.

Note that in practice we only observe the above transaction volume in terms
of monetary units rather than the full vector of prices and transacted
units. %
% These two are implicitly part of the measure $\langle\Pp,\Tp\rangle$.

\subsection{Adjustment heuristics to deflate transaction volumes}
\label{sec:particularities_txvol:adjustment_txvol}% 
%
As discussed in \refsec{concept_utxo}, only addresses but no identities are
recorded in the transaction ledger. %
This complicates the classifications of self-churn, needed to calculate
adjusted (deflated) transaction volume. %
However, statistical properties have been used to classify outputs as likely
belonging to the same individual user as the transactions inputs (compare
\cite{meiklejohn2013fistful}). %
While \emph{heuristics}, such procedures are now commonly employed to create
\emph{user clusters} of addresses. %
Outputs are classified as self-churn if the cluster of their destination
address equals the cluster of their input addresses. %

% Clustering addresses is has been done in various studies %
% (compare \cite{meiklejohn2013fistful}, \cite{spagnuolo2014bitiodine}, %
% \cite{herrera2014research} and many others). %

As our empirical analysis builds on a blockchain parser proposed
by~\cite{kalodner2017blocksci}, we follow their choice of heuristics. %
They employ one heuristic first proposed by~\cite{meiklejohn2013fistful} and
one accounting for \textit{peeling chains}. %
Peeling chains are transaction patterns where large unspent transaction
outputs are split into smaller amounts in a chain of transactions. %
Upon manual inspection \cite{kalodner2017blocksci} concluded that outputs
created and spent within a relatively short time period often belong to the
same user cluster. %
The heuristics used are thus:%
\begin{enumerate}\setlength{\itemsep}{0pt}
\item All inputs in a transaction stem presumably from one person.
  \footnote{%
    Note that \cite{kalodner2017blocksci} added the additional restriction
    that this heuristic is not applied to coinjoin transactions which are
    used to obfuscate transaction paths. %
    Coinjoin transaction are classified as in \cite{goldfeder2018cookie}.%
  } %
  % \item Addresses are only classified as change addresses if they are never reused as receipient. %
\item Outputs created and spent within 4 blocks are classified as self-churn
  transactions. %
\end{enumerate}%

%\subsection{Practical adjustment methods to deflate transaction volumes}\label{sec:particularities_txvol:adjustment_txvol}% 
%
% As our empirical analysis builds on a blockchain parser proposed %
% in~\cite{kalodner2017blocksci}, we simply adopt the default choice %
% % \footnote{\url{https://github.com/citp/BlockSci/wiki#toc-clustering-default-heuristic}} %
% of their package and form %
% clusters using the \emph{multi-input} heuristic proposed first %
% in~\cite{meiklejohn2013fistful} and an adjusted form of the \emph{optimal-change} heuristic introduced in
% \cite{nick2015data}. %
% %
% \par The multi-input heuristic assumes, that all inputs of a certain transaction can be assumed to belong to a certain user cluster. %
% While simple, the method enjoyed considerable coverage in research \citep{harrigan2016unreasonable, androulaki2013evaluating, herrera2014research} and is seen as effective measure of deanonymization \citep{frowis2019safeguarding, conti2018survey}. %
% However, the multi-input heuristic leads to unreasonable large clusters as of coinjoin transactions which are used to obfuscate transaction paths \citep{moser2017anonymous}. %
% \cite{kalodner2017blocksci} added additional restrictions %
% that this heuristic is not applied to coinjoin transactions. %
% Coinjoin transaction are classified as in \cite{goldfeder2018cookie}. %
% %
% \par The optimal-change heuristic assumes that if there is an output smaller than any of the inputs, this output can be classified as change \citep{nick2015data}. % 
% \cite{kalodner2017blocksci}, however, add the restriction that the output needs to be the first output sent to the respective address as Cryptocurrency Wallets generate fresh change addresses for change outputs. %

% As our emprical analysis builds on a blockchain parser proposed %
% in~\cite{kalodner2017blocksci}, we follow their approach to form %
% clusters using one heuristic that has been proposed first %
% in~\cite{meiklejohn2013fistful} and one heuristic accounting for %
% \textit{peeling chains}. %
% According to \cite{kalodner2017blocksci}, peeling chains are transaction %
% patterns that split large unspent transaction outputs into smaller amounts %
% in a chain of transactions. %
% Manual inspection lead \cite{kalodner2017blocksci} to the conclusion that outputs %
% that are created and spent during a relatively short time period are often %
% belonging to the same user cluster. %
% The heuristics used are thus:%
% \begin{enumerate}%
% 	\item All inputs used in a transaction are most likely from one %
%           person.%
%           \footnote{%
% 		As the basic idea leads to unreasonable large clusters, %
% 		\cite{kalodner2017blocksci} added additional restrictions %
% 		that this heuristic is not applied to coinjoin transactions %
%                 which are used to obfuscate transaction paths. %
%     	Coinjoin transaction are classified as in \cite{goldfeder2018cookie}.%
%     } %
% 	%\item Addresses are only classified as change addresses if they are never reused as receipient. %
% 	\item Outputs that are created and spent within 4 blocks, are classified as %
% 	self-churn transactions. %
% \end{enumerate}%
% Having formed user-clusters of addresses, outputs are classified as self-churn %
% if the cluster of their destination address equals the cluster of %
% their input addresses. %

\section{Velocity based on total money supply}
\label{sec:oldmeas}%
% Equipped with all necessary basics from both realms---economics and
% computer science---recently suggested methods of velocity measurement for
% cryptocurrencies are to be summarized. %
% Both approaches utilize the total coin supply as measure of the money supply
% $\Mp$ in equation \eqref{eq:fisher}. %
The quantity equation \eqref{eq:fisher} requires a measure of the money
supply $\Mp$. %
To our knowledge, all prior work has employed the sum total of ever-mined
coins as its measure. %
We denote this money-supply measure $\MTotalP$ and calculate it at the
beginning of each period $\perd$. %
\footnote{%
  As discussed in \refsec{concepts}, we simplify by assuming a fixed money
  supply over period $\perd$, using the amount at the beginning of the
  period. %
  Neither \cite{kalodner2017blocksci} nor \cite{athey2016bitcoin} clearly
  specify their adopted choice.} %
% %%% --- PLS ADD FORMULA:  (and delete "." prior to \footnote)
% as
% \begin{align}%
%   $\MTotalP$ = FORMULA.
% \end{align}
% %%% ---END PLS ADD formula.
In all common UTXO-based cryptocurrencies it is a deterministic function of
block height. %
Technically, $\MTotalP$ can be calculated as the aggregate of outputs from the set of coinbase transactions \( \TxC \) belonging to the set \( \mathbb{P} \) of all periods with a maximum block time smaller \(\perd_\Start\) and thus
\begin{align}%
  \MTotalP = \sum_{t \in \TxC} \valO(o).%
\end{align}%

Based on the quantity equation \eqref{eq:fisher}, the simplest measure of
velocity arises from dividing total transaction volume
$\langle \Pp'\Tp' \rangle$ by total coin supply $\MTotalP$, which was
described in~\cite{bolt2016value} and adopted by~\cite{ciaian2018price}. %
%
Formally,
\begin{align}%
  \VNaiveEstP = \frac{\langle\Pp',\Tp'\rangle}{\MTotalP}.%
\end{align}%
%
$\VNaiveEstP$ offers the advantages of providing a theoretically sound
interpretation and extremely simple calculation. %
Moreover, data for calculating $\VNaiveEstP$ (raw on-chain transaction volume
and total coin supply) are widely available. %
\footnote{Time-series data can be downloaded for free from %
  \url{www.blockchair.com}, \url{www.blockchain.com}, \url{www.btc.com}. %
  Also \url{www.blockwatch.cc} and many other data brokers provide the data.}
However, the result is biased: Self-churn transactions lead to an
overestimation of transaction volume. %
 
\cite{athey2016bitcoin} and \cite{kalodner2017blocksci} propose a similar
velocity measure. %
However, they clean the price sum from self-churn and get %
% For every period $\perd$ it is thus calculated as the ratio of price %
% sum~$\langle\Pp,\Tp\rangle$ divided by the complete money supply
% $\MTotalP$. %
%
% Velocity $\VTotalEst$ can thus be estimated as%
%
\begin{align}%
  \VTotalEst = \frac{\langle\Pp,\Tp\rangle}{\MTotalP}.%
\end{align}%
%
In line with \refsec{concepts}, both measures can interpreted as the turnover
of coins during period $\perd$ averaged over the total coin supply. %


%%% Local Variables:
%%% mode: latex
%%% TeX-master: "../main"
%%% End:

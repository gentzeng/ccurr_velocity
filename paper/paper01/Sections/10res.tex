% Results

\section{Benchmarking popular velocity proxies on the Bitcoin blockchain }
\label{sec:results}%

Equipped with our proposed velocity measures that are calculated on a level
of detail of each output in each transaction in each block on the blockchain,
we can now use these measures to empirically assess the quality of popular
proxies for velocity.

To this end, we first review the most common proxies in
\refsec{results:sub:approx_crypto}, then implement the proposed estimators to
calculate velocity measures for Bitcoin in \refsec{results:sub:comp}, and
finally run tests in \refsec{results:sub:goodness} to evaluate the goodness
of fit of the proxies \wrt the measures.

\ifdefined\varInputAlgos%
\renewcommand{\arraystretch}{1.5}%
\begin{algorithm*}[!h]%
	\DontPrintSemicolon
	\caption{$GetInputSpentCase$: Check whether an input $i$ was generated before window $\wndw$ or stems from a coinbase transaction.}\label{algo:code_mcirc_cond}%
	\KwData{$i$                                    	                \tcc*{Input to be checked}}%
	\KwData{$\wndw_\Start$                                    	    \tcc*{Begin of look-back window $w$}}%
	$ \inpSpentCase \gets {0} $										\tcc*{Intialize value representing spent case for input $i$}%4
	\eIf(\tcc*[f]{Check whether input $i$ stems from a coinbase transaction}){%
		$\genByCoinbase(i)$%
	}{%
		\lIf(\tcc*[f]{Check whether input $i$ was generated before $\wndw_\Start$}){%
			$\dateGen(i) < \wndw_\Start$
		}{%
			$ \inpSpentCase \gets {3} $
		}
		\lElse{%
			$ \inpSpentCase \gets {2} $
		}%
	}{%
		\lIf(\tcc*[f]{Check whether input $i$ was generated before $\wndw_\Start$}){%
			$\dateGen(i) < \wndw_\Start$
		}{%
			$ \inpSpentCase \gets {1} $
		}
		\lElse{%
			$ \inpSpentCase \gets {0} $
		}%
	}%
	return $\inpSpentCase$                                          \tcc*{Result: Return value representing spent case for input $i$}%
\end{algorithm*}%
\else%
\fi%
%%
\ifdefined\varInputAlgos%
\renewcommand{\arraystretch}{1.5}%
\begin{algorithm*}[!h]%
	\DontPrintSemicolon
	\caption{$GetCoinbaseFees$: Precompute $\Out^{\cbs}$: outputs of coinbase transactions representing collected fees.}\label{algo:code_cbs_out}%
	$\Out^{\cbs} \gets{\emptyset}$                                      \tcc*{The set of maps of coinbase tx outputs representing collected fees per blockheight}%
	\ForEach{$\mathrm{blockheight}\ bh\ \mathrm{of\ a\ given\ Blockchain}$}{%
		$\Out^{\cbs} \gets{\Out^{\cbs}\,\cup\,\{\emptyset\}}$           \tcc*{Append an empty map for this blockheight}%
		$ \outSum  \gets{0}$                                            \tcc*{Sum of regarded outputs}%
		$ \feeRem   \gets{0}$                                           \tcc*{Value of last output (partly) representing collected fees}%
		$ \minedRem \gets{0}$                                           \tcc*{Value of first output (partly) representing mined coins}%
		$ \feeSum   \gets{\sum_{f\in{}\Fee(bh)} \val(f)}$               \tcc*{Sum of all fees collected in block}%
		$ \Out'_{t}\gets{sortSpentOnly(\OutT)}$                         \tcc*{Sort ascending by indices of transactions spending the outputs and discard utxos}%
		\lIf(\tcc*[f]{Continue if all outputs are unspent}){%
			$\Out'_{t} = \emptyset$%
		}{continue}%
		\ForEach{$o \in \Out'_{t}$}{%
			$\keyOut \gets {(\addr(o),\ \val(o))}$                       \tcc*{Map key for last appended map $\Out^{\cbs}[bh]$, supposed to be unique}%
			$\Out^{\cbs}[bh][\keyOut] \gets {\indextt(o)}$  \tcc*{Add map entry, where $\keyOut$ is mapped to index of $o$ within $\Out'_{t}$}%
			$\outSum \gets {\outSum\,+\,\val(o)}$                         \tcc*{Update $\outSum$}%
			\If(\tcc*[f]{If we exceded all outputs representing fees}%
			){$\outSum \ge \feeSum$}{%
				$\minedRem \gets {\outSum\,-\,\feeSum}$                   \tcc*{Compute value of first output (partly) representing mined coins}%
				$\feeRem   \gets {\val(o)\,-\,\minedRem}$                 \tcc*{Compute value of last output (partly) representing collected fee}%
				break
			}%
		}%
		$\Out^{\cbs}[bh][\mathtt{"minedRemIndex"}] \gets {\indextt(o)}$ \tcc*{Add index of last added output $o$ within $\Out'_{t}$}%
		$\Out^{\cbs}[bh][\mathtt{"minedRem"}] \gets {\minedRem}$        \tcc*{Add value of first output (partly) representing mined coins}%
		$\Out^{\cbs}[bh][\mathtt{"feeRem"}] \gets {\feeRem}$            \tcc*{Add value of last output (partly) representing collected fees}%
	}%
%
	return $\Out^{\cbs}$                                                \tcc*{Result: The set of maps of coinbase tx outputs representing collected fees}%
\end{algorithm*}%
\else%
\fi%
%%
\vspace*{1em}%
\renewcommand{\captionGlo} {Descriptives for velocity proxies and measures
  with $\perd = \wndwLength = 1 \ttext{ day}$ for 2013-06-01 to 2019-06-01.}%
\renewcommand{\labelGlo}{\label{tbl:descriptives_est_and_app}}%
\tableTwoColumnH{descriptives_est_and_app}%
% We assume that the availability of public price data marks the beginning of
% bitcoins being used as media of exchange.
% \footnote{While the results for \ac{mae}, \ac{mse} and \ac{msc} tests hold in equivalently also for data from 2009 onwards, the Minzer-Zarnowitz regressions does not allow for conclusive interferences.}%

\subsection{Popular proxy variables for velocity of UTXO-based
  cryptocurrencies}
\label{sec:results:sub:approx_crypto}%
% The following approximations are commonly used so far.

% \subsubsection{Turnover}
% \label{sec:results:sub:approx_crypto:subsub:turn}%
% Classical \emph{turnover} refers to the simple sum total of transacted
% amounts within a period $\perd$:%
% \begin{align}
%   \TurnoverP = \sum_{t\in\TxP} \sum_{i\in\InpT} \valI(i).%
% \end{align}

% Of course, modified variants of turnover can be constructed based on
% excluding change transactions, or transactions within user clusters.

% The main advantage is its simplicity and immediate correspondence to turnover
% as understood in classical financial markets.  It lacks the scaling by the
% money supply central to the concept of velocity, but it can still be used as
% a proxy for velocity measures based on total supply---which for
% cryptocurrencies after all is locally close to flat for short periods, and
% importantly, is (in contrast to fiat currencies) exogenously determined.

\subsubsection{Coin Days Destroyed}
\label{sec:results:sub:approx_crypto:subsub:bdd}%

Since turnover can be gamed via repeated transfers of an agent to
herself,\footnote{Primarily, allegations centered on exchanges inflating
  trading volumes to pretend to be more liquid.} %
the \Ac{cdd} measure was introduced as a manipulation-proof alternative in
\url{bitcointalk.org} in 2011.%
\footnote{\url{https://bitcointalk.org/index.php?topic=6172.msg90789\#msg90789}} %
For each input, \emph{coin days} refer to the product of its monetary value
and the number of days ``since it was last spent,'' \ie how many days the
funding output had remained a UTXO.  %
\Ac{cdd} then is defined as the sum of coin days over all transactions within
a period $\perd$:%
\begin{align}%
  \BddP = \sum_{t\in\TxP} \bdd(t),%
\end{align}%
where%
\begin{align}%
  \bdd(t) = \sum_{i\in\InpT}  \Delta(i) \cdot \valI(i)%
\end{align} %
with $\Delta(i)$ denoting the number of days since the respective input
originated as output (or coinbase transaction) in a prior block, and
$ \valI(i)$ the value of input $i$ of transaction $t$ in period $\perd$.  %
% For transactions $t=1$ to $\Tp$ for period $\perd$, this can be %
% calculated using the volume of transaction $t$ including fees %
% $\val^{\tx^{+}}_\perdT$ %
% and the weighted average days $\Delta \tp_t$ since the transaction $t$ of period $\perd$ has last been spent:%
% \begin{align}\label{eq:bdd}%
%   \BddP = \sum_{t=1}^{\Tp} \Delta \tp_t \cdot \val^{\tx^{+}}_\perdT,%
% \end{align}%
% where %
% \begin{align}%
%   \val^{\tx^{+}}_\perdT %
%   = \sum_{o=1}^{O_\perdT} \val^{\out}_{\perdTi} %
%   + \val^{\mathtt{fees}}_\perdT %
%   = \sum_{i=1}^{\Ip_t} \valI(\perdTi).%
% \end{align}%
% \Ac{cdd} was initially introduced by \emph{ByteCoin} as resilient %
% measure of transaction volume, unbiased by sending money back and forth artificially. %
% However, user \emph{MoonShadow} quickly brought it into discussion as measure %
% for the velocity of money. %$

The \Ac{cdd} measure puts larger weight on the reactivation of long-dormant
coins compared to ones frequently spun, a feature clearly differing from the
concept of velocity. %

\subsubsection{Turnover based on dormancy}
\label{sec:results:sub:approx_crypto:subsub:dorm}%

In \cite{smith2017bitcoin} a proxy for velocity, aiming at ``the average number of times the actively used
[coins] can be expected to turn over'' was proposed. %
While named ``turnover'' in \cite{smith2017bitcoin}, this is not to be confused with the classical \emph{turnover} that refers to the simple sum total of transacted
amounts within a period $\perd$:%
\begin{align}
  \TurnoverP = \sum_{t\in\TxP} \sum_{i\in\InpT} \valI(i).%
\end{align}
The measure in \cite{smith2017bitcoin} in contrast is constructed as the
inverse of average dormancy $\DormP$ multiplied by the time period, where
dormancy amounts to \ac{cdd} scaled by the sum of inputs spent during the
period $\perd$ \citep[cf.][]{smith2017bitcoin}:%%
% % BEGIN SHORTENING?
% \cite{smith2017bitcoin} developed dormancy as a measure for the average time that %
% coins have been hold unused before being transacted. %
% It is calculated as the weighted average days $\Delta \tp_t$ passed since %
% the last transaction. %
% As weights, the respective transaction volumes $\val^{\tx^{+}}_{t}$ relative to %
% the aggregated transaction volumes over period $\perd$ are used:
% \begin{align}%
%   \DormP %
%   = \frac{%
%   		\sum_{t\in\TxP}\Delta \tp_t \val^{\tx^{+}}_{t}%
%   	}{%
%   		\sum_{t\in\TxP} \val^{\tx^{+}}_{t}%
%   	} .%
% \end{align}%
% The number of weighted average days $\Delta \tp_t$ since the transaction has %
% last been spend %
% can be calculated%
% \footnote{
%   In the published working paper of \cite{smith2017bitcoin} it remains %
%   unclear, how the author conceptualizes this variable exactly. %
%   We thus suggest an appropriate definition of $\Delta \tp_t$.
% } %
% from the values of transaction inputs %
% $\valI(\perdTi)$ and the days since the spent input %
% has been generated in a preceding transaction $\Delta t_{\perdTi} $ as%
% \begin{align}%
%   \Delta \tp_t %
%   = \frac{%
%   	\sum_{i\in\InpPT}\Delta t_{\perdTi} \valI(\perdTi)%
%   }{%
%   	\sum_{i\in\InpPT} \valI(\perdTi)%
%   }.%
% \end{align}%
% %
% % END SHORTENING?
\begin{align}%
  \TurnP = \frac{1}{\DormP} \cdot  \delta,%
\end{align}%
with%
\begin{align}%
  \DormP = \frac{\BddP}{\sum_{t\in\TxP} \sum_{i\in\InpT} \valI(i)}.%
\end{align}%
Here $\delta$ refers to the length of the period for which turnover is
calculated.  %
To illustrate the concept, assume the coins spent today stayed unused on
average for 6 hours before their transaction.  %
Hence, circulating coins are turned over
$\frac{\SI{24}{\hour}}{\SI{6}{\hour}} = 4$ times per day on average.  %
Dormancy-based turnover, however, remains an approximation depending on
transactions distributed homogeneously over time.  %
% To see that, consider an example where there is only one coin %
% that is spend $5$ times rapidly over and over again in the first minutes %
% of the day. After that, nothing happens. The inferred approximation could be arbitrarily high depending on how little time passes %
% between the transactions. %
\renewcommand{\labelGlo}    {\label{fig:globfigComponents}}%
\renewcommand{\captionLocL} {Components of velocity measures.}%
\renewcommand{\captionLocR} {Correlations between the measures and price.}%
\renewcommand{\labelLocL}   {\label{fig:subfigComponentsPlot}}%
\renewcommand{\labelLocR}   {\label{fig:subfigComponentsCorr}}%
\figureTwoColumn{ts_figs/desc_components}{ts_figs/corrplot_components}%
\renewcommand{\captionGlo}  {Time series plots for proxy-variables.}%
\renewcommand{\labelGlo}    {\label{fig:globfigApp}}%
\renewcommand{\captionLocL} {Normalized variables of volatility approximation.}%
\renewcommand{\captionLocR} {Standardized variables of volatility approximation.}%
\renewcommand{\labelLocL}   {\label{fig:subfigAppNorm}}%
\renewcommand{\labelLocR}   {\label{fig:subfigAppStand}}%
\figureTwoColumn{ts_figs/desc_all_app_norm}{ts_figs/desc_all_app_stand}%

\subsection{Data sources and calculations}%Origin and description of approximation and measurement data}
\label{sec:results:sub:comp}%

%%% First data:
We collect a dataset spanning from June 2013 until June 2019, starting with
the rise of the first cryptocurrency exchanges and thus reliable trading
data. %

For the proxy variables, we tap existing sources as far as possible.  %
\Ac{cdd} data is gathered via API from
Blockwatch,\footnote{\url{https://blockwatch.cc}} %
trading data taken from
CoinMarketCap.\footnote{\url{https://coinmarketcap.com}}  %

To calculate the velocity measures, however, access to the atomic units of
cryptocurrency transactions is needed.  %
We rely on the open-source blockchain parser \emph{BlockSci} introduced by
\cite{kalodner2017blocksci}.  %
We also provide the code of our paper at
\url{https://github.com/wiberlin/ccurr_velocity/}. %
In replicating the clustering approach of \cite{kalodner2017blocksci} (see
\refsec{particularities_txvol:adjustment_txvol}), in order to mitigate the
amount and effect of false positives we exclude one unreasonably large
cluster with over 297 million addresses.  %
\footnote{Only 13 clusters contain more than $20,000$ addresses.}  %

%%% Then calculations:
We calculate our velocity measures based on a time window for money in active
circulation $\wndwLength$ equal to period $\perd = 1 \text{ day}$.%
\footnote{We define days based on transaction block times and UTC
  timezone.} %
Therefore, a \textit{coin} is in circulation if it is transferred at least
once within the day for which velocity is computed.  %
This daily measure can be interpreted as average turnover of monetary units
which are part of the daily circulating money supply.  %
While our choice for the window $\wndwLength$ within which moved coins are
considered actively in circulation is short, this allows us to flesh out the
difference between velocity based on the total money supply and our proposed
measures based on circulating supply most clearly.  Moreover, since
\cite{kalodner2017blocksci} provide results with an
implicitly\footnote{\cite{kalodner2017blocksci} equate money in circulation
  with total coin supply, implying an infinite time window.}  %
infinite $\wndwLength$, we provide evidence on the opposite endpoint on the
spectrum.  %
Naturally, neither the BlockSci parser nor our approach are restricted to
these choices and can be extended both to other time windows and other
UTXO-based cryptocurrencies.  %
% Calculating a daily velocity, for example, it might be interesting to base %
% the measure on money that circulated within the last week, month or year. %
% Analyzing the impact of the time window selection on velocity and its %
% relationship to returns or volatility might be highly interesting but %
% is left for future research. %
% In this section, we will display central characteristics of %
% approximation and estimator data and compare them from different perspectives. %

%%% Then properties of the calculated numbers:
\reftbl{descriptives_est_and_app} shows descriptive statistics for the
proxies and the measures.  %
$\Bdd$ denotes \ac{cdd} in million coin days, while $\Turn$ denotes
\emph{active turnover} in expected on-chain coin transfers.  %
For both proxy variables, means exceed medians, suggesting outliers in the
skewed distribution.  %
% 
According to $\VNaiveEst$, a monetary unit of the total coin supply is turned
over $0.11$ times on average, while the more sophisticated measure
$\VTotalEst$ results in turnover of $0.09$.  %
The difference stems from the deflated transaction volume used by
$\VTotalEst$ being strictly lower then the inflated one (see
\refsec{particularities_txvol}).  %
According to the measure $\VCircWbEst$, which is based on the \ac{wba}, coins
in effective circulation during the day reach turnover of around $5.83$.  %
Assuming the clustering heuristics work well, coin transfers correspond to
peer-to-peer hops.  %
Accordingly, $\VCircWbEst$ estimates that monetary units in circulation
change owners $5.83$ times per day, while $\VCircMfEst$ and $\VCircMlEst$ give
an estimate of around $6.09$ peer-to-peer hops.  %
The reason for the higher turnover estimate is the more conservative
operationalization of the concept of \textit{being in circulation} (see
\refsec{cc_money_seg:sub:mcirc_pract}).  %

The different levels can be disentangled by looking at the components of the
velocity measures.  %
\reffig{subfigComponentsPlot} exemplarily shows this for the components of
$\VTotalEst$ and $\VCircWbEst$.  %
While $\MTotal$ increases steadily over time, the subset of coins transacted
at least once per day, with an average \SI{1.5}{\percent} of the total
supply, is minuscule but volatile in comparison.%
\footnote{Compare appendix \ref{tbl:appendix_desc_other},
  \ref{tbl:appendix_desc_other_norm}, \ref{tbl:appendix_desc_other_stand} and
  \ref{tbl:corrtable_all} for additional descriptive statistics on the
  dataset.}  %
The deflated on-chain transaction volume varies widely---clearly always below
total supply, yet above the supply in circulation.  %
While the volatility in the transaction volume feeds fully into $\VTotalEst$,
for $\VCircWbEst$ the relation is less obvious.  %
In \reffig{subfigComponentsCorr}, the components' co-variation with bitcoin
price indicates that not only deflated on-chain transaction volume
$\mathtt{Vol}$, but also the monetary aggregates are positively correlated
with price changes.  %
% While this provides only anecdotal evidence, it is consistent with the
% hypothesis that transaction volumes as well as the liquid component of the
% money supply rise with price, and in consequence lower the correlation of the
% velocity measure $\VCircWbEst$ with price.  %
% This hypothesis is related to the broader question how far speculation might
% act as ``regulator of the quantity of money'' \citep{fisher1911equation}.  %
Somewhat surprising the velocity measures show a slightly negative correlation
with prices however. %
% 
\renewcommand{\captionGlo}{Components of velocity measures
  $\protect\VTotalEst$ and $\protect\VCircWbEst$ with
  $\perd = \wndwLength = 1 \ttext{ day} $.}%

Comparing maxima and minima across time series, as
\reftbl{descriptives_est_and_app} shows, requires scaling.  %
We use two methods: normalization and standardization.  %
Normalization is based on the usual %
$X^{norm} = (X-X_{min})/(X_{max}-X_{min})\in[0, 1]$.  %
Standardization is based on Z-scores
$X^{stand} = \frac{X-\mu(X)}{\sigma(X)},$ with mean $\mu(\cdot)$ and standard
deviation $\sigma(\cdot)$.  Both are sensitive to outliers
\citep[cf.][]{angelov2019empirical}, so we truncate at $10$ standard
deviations around the mean.%
\footnote{This procedure truncated a maximum of 3 values across time
  series.}  %
%
\reffig{globfigApp} shows the time series of proxy variables for velocity.  %
The scaling leads to a visible difference, relevant when comparing proxies to
measurement results.  %
A first indication of the quality of the proxy variables is their
diversity.  %
Not only spikes but also general trends vary across methods.  %
%
\reffig{globfigEst} depicts the different velocity measures.  %
Differences across measures are smaller, highs and lows correspond more. %
% The figure further shows the similarity between \ac{wba} and \ac{mca}. %
The next section provides quantitative evidence.

\renewcommand{\captionGlo}   {Time series plots for volatility measures with $\perd = \wndwLength = 1 \ttext{ day} $.}%
\renewcommand{\labelGlo}     {\label{fig:globfigEst}}%
\renewcommand{\captionLocL}  {Normalized variables of volatility measurement.}%
\renewcommand{\captionLocR}  {Standardized variables of volatility approximation.}%
\renewcommand{\labelLocL}    {\label{fig:subfigEstNorm}}%
\renewcommand{\labelLocR}    {\label{fig:subfigEstStand}}%
\figureTwoColumn{ts_figs/desc_all_est_norm}{ts_figs/desc_all_est_stand}%

\subsection{Assessing the goodness of fit of the proxies}
\label{sec:results:sub:goodness}%
We now turn to evaluate the proxy variables for velocity based on our
estimated measures.  %
In so doing, we include the trivial velocity measure $\VNaiveEst$ in the set
of proxies, as it is calculated on very high-level inputs (raw on-chain
transaction volume and total coin supply) that are similarly easy to obtain
as \ac{cdd} or \emph{turnover}.%
% \footnote{Note that \ac{cdd} for Bitcoin are available at 
%   many free blockchain explorers. \emph{Turnover} can, in combination with the 
%   daily inflated transaction volume, be inferred from the latter.} 

In order to evaluate the quality of proxy variables perfectly, the ``true''
velocity ought to be known.  At the same time, we have proposed three novel
measures, in addition to those suggested by \cite{athey2016bitcoin} and
\cite{kalodner2017blocksci}.  We do not view this a contradiction: after all,
the measures capture different concepts of velocity.  Ours, for example,
address the velocity of money in circulation.  Therefore, we do not horserace
all proxies against one benchmark, but rather evaluate the goodness of fit of
the proxy variables with any of the measures.  We do, however, take the
position that the measures are more precise estimators of ``true velocity''
(its respective concepts) than the proxies, and thus evaluate the latter
in terms of their fit to the former.

Perhaps surprisingly, the results do not vary qualitatively much across
different measurement approaches. %, so our results do not depend on a certain
%velocity measure.  %

\subsubsection{Approximation errors}
\label{sec:results:sub:comp:subsub:errors}%
A common approach assesses approximation errors by comparing \aclp{mse} or
\aclp{mae}.  %
\ac{mse}, by squaring them, punishes large deviations more rigorously than
the linear \ac{mae}.  We provide both, not only to the standardized and
normalized time series, but also on their first differences.  %
This is motivated, as in econometric studies like those in \refsec{lit}, by
doubts about the stationarity of the time series.  %

\reftbl{errortable} shows that the above transformations differ in their
assessment of the goodness of fit of approximation methods.  %
When judging by the normalized but undifferenced dataset, \emph{turnover}
$\Turn$ (as alternatively defined by \cite{smith2017bitcoin}) achieves the
lowest error in approximating velocity measures based on effectively
circulating money supply.  %
However, for all other constellations we find that the trivial measure
$\VNaiveEst$ provides the closest fit to all ways to measure velocity.  %


\subsubsection{Model confidence set test}
\label{sec:results:sub:comp:subsub:mcs}%
To understand if the approximation methods indeed differ significantly, we
perform \acf{mcs} tests.  %
\cite{hansen2011model} introduced \Ac{mcs} tests to measure the performance
of forecasting methods, but they are applicable more generally.  %
The method uses equivalence tests and an elimination procedure to determine
which set of models significantly outperforms the rest of the models.  %
For an intuitive understanding of the test one might think of a set $M$ of
models (here the different approximation methods).  %
The models are indexed using $i \text{ and } j \in 1,..,m $.  %
When comparing model $i$ to a benchmark, for each period $\perd$ model $i$
leads to loss functions
$ L_{i \perd} = L(V^{\est}_\perd, V^{\app}_{i \perd})$, where
$V^{\est}_\perd$ denotes the benchmark velocity measure and
$V^{\app}_{i \perd}$ the approximation method $i$.

To compare the performance of the models in $M$, relative performance metrics
$d_{ij \perd}$ are defined as %
\begin{align*}
  d_{ij \perd} = L_{i \perd} - L_{j \perd}%
\end{align*}
for all $ i,j \in M $ where $i \neq j $.  %
We use absolute and squared errors for the loss function $L$, so that it takes the forms %
\begin{align*}
  d_{ij \perd} = \bigl| V^{\est}_\perd
  - V^{\app}_{i \perd} \bigr|
  - \bigl| V^{\est}_\perd
  - V^{\app}_{j \perd} \bigr|%
\end{align*}
or
\begin{align*}
  d_{ij \perd} = \bigl( V^{\est}_\perd
  - V^{\app}_{i \perd} \bigr)^{2}
  - \bigl( V^{\est}_\perd
  - V^{\app}_{j \perd} \bigr)^{2}%
\end{align*}
respectively. %

The relative performance of model $i$ compared to all other models then is
\begin{align*}
  d_{i \boldsymbol{\cdot}} = \frac{1}{m-1} \sum_{j \in M \setminus i} d_{ij},%.
\end{align*}
with $i = 1,...,m$. %
The null hypothesis states %
\begin{align*}
  H_{0M}: E(d_{i \boldsymbol{\cdot}}) = 0,\ \forall i \in M. %
\end{align*}
If it can be rejected for set $M$, there exist models in the set that are
significantly outperforming the remaining ones.  %
The above equivalence test is then re-iterated after elimination of the
worst-performing model(s) until the null hypothesis cannot be rejected.  %
For details, see \cite{hansen2011model}.

%\FloatBarrier
\renewcommand{\captionGlo} {%
  Mean absolute error for normalized data of approximation methods compared %
  to measurement methods with $\perd = \wndwLength = 1 \ttext{ day} $. %
  Proxy-variables in \SI{1}{\percent} \acl{mcs} marked by $\dag$.%
}%
\renewcommand{\labelGlo}{\label{tbl:errortable}}%
\tableTwoColumnF{appVSest_errors_w_mcs}
%
\addtolength{\tabcolsep}{-2pt} \renewcommand{\captionGlo} {\acl{mcs}
  regressions for standardized and normalized approximation and measurement
  data with $\perd = \wndwLength = 1 \ttext{ day}$.}%
\renewcommand{\labelGlo} {\label{tbl:mzregression}}%
\tableTwoColumnF{appVSest_mz_table} \addtolength{\tabcolsep}{+2pt}
%\FloatBarrier


\reftbl{errortable} displays the \ac{mcs} tests' results, denoting
significance at the \SI{1}{\percent} level with $\dag$.  %
Results for the differenced dataset are clear: the \ac{mcs} tests select even
at the tight \SI{1}{\percent} significance levels a single winning
approximation method for each constellation.  %
This evidence confirms \refsec{results:sub:comp:subsub:errors} at high
significance levels.  %
$\VNaiveEst$ is the single best element of the \SI{1}{\percent}-\ac{mcs} for
all constellations.  %
For the normalized undifferenced dataset the \emph{turnover} $\Turn$ is
the single best proxy for velocity based on circulating money for all but one
of the constellations. %
If velocity is measured as $\VTotalEst$, however, the trivial measure
$\VNaiveEst$ again performs significantly better than all other proxies.  %
In summary, \ac{mcs} tests mostly support the trivial measure
$\VNaiveEst$.  %
Only in cases with time trends and less focus on outliers than on smaller
changes, \emph{turnover} might be the better choice.  %
Again, $\Bdd$, the most common proxy, is significantly outperformed in all
constellations.  %


\subsubsection{Mincer-Zarnowitz regressions}\label{sec:results:sub:comp:subsub:mzr}%
The \ac{mz} approach regresses estimates on the benchmark in simple
ordinary-least-squares regressions \citep{mincer1969evaluation}.  %
In order to avoid spurious regression results, we only use the first
differences of the standardized and normalized series.  %
The regression structure for proxy $i$ can be expressed as:
\begin{align*}
  \Delta V^{\est}_{i \perd} = \alpha_{i} + \beta_{i} \Delta V^{\app}_{i
  \perd} + u_{i \perd}.
\end{align*}

An ideal proxy would yield an intercept of 0 and a $\beta$ equal to 1 with an
adjusted $R^{2}$ of $1$.  %
The results of the \ac{mz} regressions in \reftbl{mzregression} confirm the
simple ratio of inflated on-chain transaction volume to total coin supply
$\VNaiveEst$ as superior.  %
This simple ratio yields an adjusted $R^{2}$ between $0.34$ and $0.53$; other
proxies' $R^{2}$ is much lower for all measurement methods of velocity.  %
While for none of the approximations a significant intercept indicates bias,
the slope coefficients $\beta$ for most constellations are significant and
positive.  %
Furthermore, with respect to the $\beta$ coefficients, $\VNaiveEst$ performs
best.  %
The latter shows coefficients between $0.99$ and $1.07$ for normalized and
between $0.78$ and $0.80$ for standardized data.  %

The \ac{mz} regressions are in line with the evidence in prior sections.  %
Again, the trivial measure approximates the more sophisticated velocity
measures better than the two commonly used proxy variables.  %



%\subsection{The robustness of V-mcirc to hype-cycles}\label{sec:results:sub:comp_hype}%
%\subsection{The "moneyness" of Bitcoin derivatives}\label{sec:results:sub:comp_moneyness}
%\section{Outlook: Off-chain measures}\label{sec:results:sub:offchain}%
% Already in 2013, a considerable amount of transactions was taken off-chain %
% to exchanges~\cite{glaser2014bitcoin}. While there are no later studies on %
% the relation of on- to off-chain volume we suspect that the fraction of %
% off-chain transaction increased with the boom cycle in 2017 and large %
% scale mainstream media coverage.
% 
% We thus feel the need to calculate the velocity of money for BTC hold %
% on exchanges. % 
% Using again the transposed Fisher Equation to calculate velocity %
% off-chain BTC possessions, define $M_{off-chain}$ as BTC hold by blockchain %
% addresses belonging likely to exchanges or wallet providers. We then extract %
% $(PT)_{off-chain}$ as the trading volume in BTC on the respective exchanges %
% over the given time period. %
%
%%% Local Variables:
%%% mode: latex
%%% TeX-master: "../main"
%%% End:
%
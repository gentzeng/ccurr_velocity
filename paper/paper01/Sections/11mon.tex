% %Moneyness section
% \section{Discussion: Missing off-chain transactions---bug or feature}\label{sec:moneyness}%
% %
% While our analysis covered a lot of different aspects of on-chain velocity of %
% money for cryptocurrencies, obviously a conceptional weakness is the economic %
% importace of off-chain transactions on exchanges.%
% \footnote{Off-chain trading volume was roughly XYZ times higher then the corresponding on-chain volume end of 2018.} %
% In this section, we first propose a potential way to include off-chain transactions into the estimation of velocity but would like to add a different use case of velocity turning the exclusion of investment and trading into a desirable feature. %
% %
% \subsection{Including off-chain transactions}\label{sec:moneyness:offchain_idea}%
% A potentially viable approach to measure on-chain velocity might be to identify public addresses controlled by major exchanges%
% \footnote{This has been done e.g. by \cite{meiklejohn2013fistful}.} %
% and measuring the amount of money hold on these addresses over time. %
% While this aggregate might be interpreted as money supply available for off-chain transactions, the simple off-chain transaction volume of the exchange might represent an the transaction volume. %
% A the ratio of the two might be considered as first estimate of an off-chain velocity of money. %
% This estimate, in turn, might be aggregated with our on-chain measures by building a weighted sum using the involved amounts of money supply as weights. %
% Remaining a rough scetch, however, this is left to future research. %
% %
% \subsection{Velocity and a cryptocurrencies usage as medium-of-exchange}\label{sec:moneyness:measure}%
% To the best of our knowledge, prior economic and technical cryptocurrency %
% studies applied the concept of velocity exclusively as simple regressor %
% in empirical analyses of price determinants or part of theoretic pricing %
% models (compare section~\ref{sec:lit}). %
% We would argue, however, that with its convenient interpretation of average %
% turnovers of monetary units within a certain time period, the discussed %
% measures might already be seen as a first step towards %
% measuring a cryptocurrencies usage as medium-of-exchange and thus one of %
% the central functions of money. %
% \par Many pricing models for cryptocurrencies%
% \footnote{Compare \cite{schilling2018dp12831}, \cite{athey2016bitcoin}, \cite{garratt2018bitcoin} or \cite{biais2018equilibrium}} %
% assume, that expectations %
% of future adoption as medium-of-exchange drives demand. %
% A good measure might help to quantify trends of adoption and form more %
% educated expectations in cryptocurrency pricing models. %
% Which of the two grant types of velocity estimation is suited better? %
% Both, broadly speaking, calculate an estimate for an ``average coin turnover''. %
% While $\VCircP$ averages only over money in effective %
% circulation, $\VTotalEstP$ averages over the complete money %
% supply and thus reflects flows between the money components to a larger degree. %
% \par This difference leads to an interesting distinction in the reaction to hype cycles. % 
% Induced by the expectation of a "boom" in cryptocurrency prices, monetary units %
% in circulation might be kept in hot- or cold-wallets to wait for future price %
% increases or be kept on exchanges for trading. %
% Expecting a \textit{bust}, cryptocurrency holders might reactivate money units in %
% their wallets to transfer them to exchanges for selling them off %
% (compare e.g. \cite{glaser2014bitcoin,gurdgiev2018ripples,celeste2018fractal}). %
% % These activities could be seen as \textit{portfolio management} activities which should not effect a
% % measure for the degree to that a cryptocurrency is used as medium of exchange. %
% While there has little research been done yet, exchanges might be %
% expected to process most of trading off-chain and hold a pooled %
% amount of cryptocurrency units on their own wallets to save costs. %
% Indication is provided by high off-chain trading volumes compared to %
% on-chain transaction volumes and balance sheets %
% (compare \cite{anderson2019bitcoin}). %
% Therefore, the aforementioned activities on-chain might be reflected by %
% UTXOs staying unspent for a long time, being transferred once and yielding %
% a new UTXO which stays again unspent for a long time.%
% \par In the measure $\VTotalEstP$, these activities are imported %
% via the changing fraction of money with a velocity of zero. %
% As most of the money supply is dormant, even ``sleep-hop-sleep'' transaction %
% patterns as the ones described above increase the aggregated measure. %
% As a consequence, for an increase in velocity $\VTotalEstP$ we cannot be sure, whether monetary %
% units in circulation achieve more turnovers in a certain period $\perd$ or %
% whether just more monetary units have been turned over exactly once due to speculation.%
% \footnote{While this is note only way speculation might represent itself on-chain, %
%   we argue that the described pattern covers a significant part of scenarios.} %
% %
% $\VCircP$, however, is always based exactly on the monetary stock in circulation. %
% The described ``sleep-hop-sleep'' patterns contribute to an increase of both of %
% its components: money supply \( \MCircP \) and transaction volume $\langle\Pp,\Tp\rangle$. %
% This causes the measure to handles hype-driven transaction patterns in a more graceful way. %
% Additional money in circulation only increases the measure, when the monetary %
% units are used for an above average intensity of peer-to-peer transactions. %
% If not, the measure even might decrease as reaction to additional money in %
% circulation that is used in a below-average manner. %
% While additional research is encouraged, we form the hypothesis that \(\VCircP\) might %
% be a preferable measure when using velocity of money as approximation %
% of a cryptocurrencies usage as medium of exchange. %
% %Future research might analyse to which degree the measure drives long term prices levels. %

% % Speculation has been shown to interact with the component of money hold for %
% % the purpose of long-term investment. %
% % \cite{smith2017bitcoin} shows that higher prices correlate strongly with %
% % larger time periods since the transacted monetary supply as been last used %
% % in transactions before. %
% % \textbf{[Missing diagram [2.cW1, 2.eT1, 2.eT2]]} relates price lagged daily %
% % price changes to \ac{cdd} for Bitcoin, Bitcoin Cash, Litecoin, %
% % Namecoin and Dash. %
% % Similar to \cite{smith2017bitcoin}, we find that ...
% % \textbf{[Missing diagram [3.cW1, 3.eD1, 3.eD2, 3.eT1, 3.eT2]]} gives the %
% % number of transactions with outputs belonging to addresses of major %
% % exchange-wallets  (Kraken, Coinbase and Bitfinex) relative to lagged %
% % daily price changes. %
% % The dotted line highlights the number of transactions that have stayed %
% % at this address for over one month. %
% % \footnote{%
% %	The respective addresses have been gathered from %https://www.walletexplorer.com.%
% %} %
% %
% % Among others, \textbf{[Citation: Corbet 2017, Corbet 2018]} found existence %
% % for pricing bubbles for Bitcoin and Ethereum,  Cheah and Fry 2015 and 2016 %
% % tested for the existence of bubbles in Bitcoin prices.
% % Glaser et al 2014 find that the motivation of users, exchanging fiat %
% % currency to Bitcoin is driven by demand for a speculative asset, rather %
% % than as payment means. Mai et al 2015, Fry 2018 and  find impact of trends %
% % in posts in social media channels on prices. %
% % Off-chain speculative transaction are reflected only slightly in on-chain %
% % transactions. %
% % On the one hand, this is indicated by the enormous gap between on-chain and %
% % off-chain trading volume and on the other by balance sheet analyses of large %
% % exchanges. %
% % 
% % Speculation on-chain is to large degree represented by UTXOs staying unspent %
% % for a long time, being transferred once and yielding an new UTXO which stays %
% % unspent again for a long time. 
% % \footnote{Obviously, these components are, apart from their turnover rates, %
% % in principle indistinguishable (compare \cite{likitkijsomboon2005marx}).}
% %



% %%% Local Variables:
% %%% mode: latex
% %%% TeX-master: "../main"
% %%% End:

%Segregation of money

\section{Novel measures of velocity of money in effective circulation}
\label{sec:cc_money_seg}%

Based on the concept of velocity of money in effective circulation as
clarified last section, we now propose algorithms to calculate velocity for
UTXO-based cryptocurrencies. %

\subsection{Determining money in circulation}
\label{sec:cc_money_seg:sub:mcirc_concept}% 

Conceptually, a monetary unit has circulated in a period if and only if it
has been used as a medium of exchange.  Therefore, first a time window must
be specified with respect to which monetary units are to be classified as
\emph{circulating} or not.  %
Money is referred to as circulating if it has been moved economically within
the last day, month, year or generally any time period $\wndw$ covering
$[\wndw_\Start , \wndw_\End]$. %

To identify the fraction of the money supply which circulated within $\wndw$,
we step trough every transaction recorded in period $\wndw$.  %
Transactions spending outputs generated before $\wndw$ as well as outputs
from coinbase transactions (seignorage) have the interpretation of bringing
an amount into circulation that corresponds to the value of spent outputs.  %
%%% NOTE: Why would fees be excluded here?? -> DISCUSS %%%hwe
All inputs referring to UTXOs generated \emph{within} period $\wndw$, on the
other hand, re-spend money which has already been counted as circulating.%
\footnote{To develop an intuition, remember the example of sheep Eve being
  sold and re-bought by Bob in $2020$.  %
  This time Bitcoin was used and Alice held unspent outputs of $2$ bitcoin
  from transactions $5$ years ago.  %
  The transaction formed by Alice to buy sheep Eve, $Tx_1$, spends her
  five-year-old output and generates a new one in 2020 which is controlled by
  Bob.  %
  When Bob buys Eve back, he pays with the output of $1$ bitcoin from $Tx_1$
  in a second transaction ($Tx_2$).  %
  For $\wndw = \perd = 2020$, the value of $Tx_1$ (1 bitcoin) originated
  prior to 2020 and is thus money brought into circulation.  %
  $Tx_2$, however, spent an output generated in 2020---an already counted
  monetary unit.  %
%In the example, thus $\MCirc_{2019}=1 \ttext{ Bitcoin}$. 
} %
% This can be illustrated with \reffig{mcirc_concept} where values are %
% symbolized with coins. %
% Here, we would need to determine, how many monetary units have made the %
% transaction volume $8$ (sum of outputs in C, D, E and F excluding
% self-churn) during period $\wndw$ possible. %
% In this example, we would need to focus on transactions A, B and D in
% contrast to transactions C and E which reused monetary unspent transaction
% outputs that have been generated within period $\wndw$. %

Note that we define the time window $\wndw$ within which spent coins are
considered as \emph{circulating} distinct from the period $\perd$ for which
the velocity measure is calculated, $ [\perd_\Start, \perd_{\End}]$.  This
distinction can be parametrized via a maximum length $\wndwLength$ of the
look-back window $\wndw$, where $\wndw_\Start = \perd_\Start - \wndwLength$
and $\wndw_\End = \perd_\End$.  %

The approach as characterized so far, however, does not account for two
technical properties of UTXO-based cryptocurrencies: %
First, transactions always spend prior transaction outputs in full.  %
Second, there exists no attribution of individual outputs to the inputs of a
transaction; thus it remains undefined which input corresponds to which
output(s).  %

The first property raises the question how to deal with transactions which
send back change: %
Should the sum of all inputs be considered \textit{in circulation,} as
technically all was transferred---or should only the fraction sent to third
parties be considered?  %
We analyze both choices, naming the first \ac{wba} and the second
\ac{mca}.  %
They are visualized in \reffig{mcirc_concept}.  %
The moved-coin approach considers only output $\mathsf{Out_1}$ of transaction
$\mathsf{Tx_C}$ as circulating, not the change output.  %
This approach captures the net economic value transferred to a third
party.  %
The \ac{wba} classifies the whole input of transaction $\mathsf{Tx_C}$ as
circulating.  %
This approach captures the amount of money that has been moved technically;
it can also be interpreted as revealed to be available for transactions.  %

In the \ac{mca}, the sum of inputs is counted as circulating only net of
change outputs.  %
However, due to the second property of UTXO blockchains ambiguous
constellations can occur: If for a given transaction one input was generated
within and one before $\wndw$, it remains unspecified which one corresponds
to the change output.  %

Transaction $\mathsf{Tx_F}$ in \reffig{mcirc_concept} illustrates the
point.  %
It has two inputs: $\mathsf{Inp_1}$ originated before period $\wndw$ and
$\mathsf{Inp_2}$ originated within $\wndw$.  %
If only amounts sent to third parties matter, it remains unclear which of the
two inputs funded the change output.  %
For $\mathsf{Inp_1}$, the transaction would not increase money in
circulation.  %
In contrast, if $\mathsf{Inp_2}$ funded the change output, the transaction
would increase the amount in circulation by $\mathsf{Inp_1}$. %

To resolve the ambiguity, an assignment rule between transaction inputs and
outputs is required.  %
% Utilizing the terminology of cost accounting,
We consider both endpoints on the spectrum of the age of the input assigned
to the change transaction and thus differentiate between \ac{lifo}, where
oldest inputs get assigned to outputs first, and \ac{fifo}, where it is the
other way around.  %

Naturally, with the \ac{wba} this differentiation is void.  %
Hence, we have three definitions of money in circulation, each a function of
the activity window length $\wndwLength$: %
Money in circulation for period $\wndw$ adopting the \ac{wba}
($ \MCircWbPWl $), and both the \ac{mca} with the \ac{lifo} rule
($ \MCircMlPWl $) and the \ac{fifo} rule ($ \MCircMfPWl $).

\subsection{Definition of velocity measures}
\label{sec:cc_money_seg:sub:}%

Based on the above definitions of money in circulation, three variations of
velocity can be calculated in accordance with \refequ{vcirc_concept}---one
per money aggregate.  %
%
All measures capture the average number of peer-to-peer coin turnovers of
effectively circulating monetary units in period $\perd$.  They concur in
capturing on-chain liquidity; they differ \wrt to definitions of
\textit{circulating monetary units} and assignment rules linking transaction
inputs to outputs.  %
% 
The first measure is based on $\MCircWbP$ and simply calculated as %
\begin{align}%
  \label{equ:VCircWbEstPWl}
  \VCircWbEstPWl = \frac{\langle\Pp,\Tp\rangle}{\MCircWbEstPWl}.%
\end{align}%
$\VCircWbP$ is recommended if a conservative measurement of coin turnover is
sought or additional assumptions about how to link inputs to outputs should
be avoided.
% 
The second and third measures are based on $\MCircMfPWl$ and $\MCircMlPWl$,
respectively:  %
\begin{align}%
  \label{equ:VCircMfEstPWl}
  \VCircMfEstPWl = \frac{\langle\Pp,\Tp\rangle}{\MCircMfEstPWl}, \\
  \VCircMlEstPWl = \frac{\langle\Pp,\Tp\rangle}{\MCircMlEstPWl}.%
  \label{equ:VCircMlEstPWl}
\end{align}%
%
$\VCircMfEstP$ and $\VCircMlEstP$ are more stringent on the definition of
money in circulation.  %
Their monetary aggregates do not count ``touched'' funds, but only amounts
transferred to somebody other than the sender.
% Another advantage of the two measures is lower boundedness at 1 if \(\perd\) is set equal to \(\wndw\) which allows for
% an intutive understanding of the circulation of coins.%
% \footnote{Every transaction, by construction, can draw at most its own
%   transaction value into circulation. %
%   Transaction volume in both, numerator and denominator, is calculated free of change money.} % 
However, this comes at the cost of the additional assumption with respect to the assignment rule between transaction inputs and outputs.  %

\subsection{Algorithmic implementation}
\label{sec:cc_money_seg:sub:mcirc_pract}%
Having defined three velocity measures, we now detail our technical approach
and the implementation.

Money in circulation under the \ac{wba} is measured as in
\refalgo{code_mcirc_wb}.  %
For every period $\wndw$, we loop over all transactions $t\in\TxW$ and add
their inputs to circulating money if they either reference outputs from
coinbase transactions, denoted by $\genByCoinbase(i)$, or outputs with
timestamps $\dateGen(i)$ before the first timestamp $\wndw_\Start$ of period
$\wndw$.  %
%
%  \footnote{In \reffig{mcirc_concept}, the algorithm would go trough %
%  transaction C (adding output $\mathsf{1}$ of transaction B, referenced by input $\mathsf{1}$ %
%  of transaction C), would then go to transaction D (finding no input at all), %
%  then to transaction E (skipping output $\mathsf{1}$ of transaction E but adding output $\mathsf{1}$ %
%  of transaction D as referenced by input $\mathsf{2}$ of transaction E) and so forth. %
% } %
%
%\ifdefined\varInputAlgos%
\renewcommand{\arraystretch}{1.5}%
\begin{algorithm*}[!h]%
	\DontPrintSemicolon
	\caption{$GetInputSpentCase$: Check whether an input $i$ was generated before window $\wndw$ or stems from a coinbase transaction.}\label{algo:code_mcirc_cond}%
	\KwData{$i$                                    	                \tcc*{Input to be checked}}%
	\KwData{$\wndw_\Start$                                    	    \tcc*{Begin of look-back window $w$}}%
	$ \inpSpentCase \gets {0} $										\tcc*{Intialize value representing spent case for input $i$}%4
	\eIf(\tcc*[f]{Check whether input $i$ stems from a coinbase transaction}){%
		$\genByCoinbase(i)$%
	}{%
		\lIf(\tcc*[f]{Check whether input $i$ was generated before $\wndw_\Start$}){%
			$\dateGen(i) < \wndw_\Start$
		}{%
			$ \inpSpentCase \gets {3} $
		}
		\lElse{%
			$ \inpSpentCase \gets {2} $
		}%
	}{%
		\lIf(\tcc*[f]{Check whether input $i$ was generated before $\wndw_\Start$}){%
			$\dateGen(i) < \wndw_\Start$
		}{%
			$ \inpSpentCase \gets {1} $
		}
		\lElse{%
			$ \inpSpentCase \gets {0} $
		}%
	}%
	return $\inpSpentCase$                                          \tcc*{Result: Return value representing spent case for input $i$}%
\end{algorithm*}%
\else%
\fi%
%

Measuring money in circulation under the \ac{mca} is depicted in
\refalgo{code_mcirc_mc}.  %
%
As in \refalgo{code_mcirc_wb}, for time window $\wndw$ we loop over all
transactions $t\in\TxP$ and add inputs based on the same core condition
(compare lines~\ref{algo:code_mcirc_mc}-\ref{algo:code_mcirc_mcCond} and
\ref{algo:code_mcirc_wb}-\ref{algo:code_mcirc_wbCond}).  %
This time, however, only those inputs are regarded for further counting,
which add up to the amount sent to third parties. Therefore, the calculated
amount of money in circulation per transaction can be less or equal to the
amount sent to third parties, but never more.
The order of inputs to consider is determined by the \ac{lifo} or \ac{fifo}
principle.  %
For every transaction $t$ in time window $\wndw$, the amount sent to third
parties is determined net of self-churn as
$\valO^\sendToOthers(t) = \sum_{o\in{}\Out'_{t}} \valO(o)$ %
where $\Out'_{t}$ denotes non-self-churn outputs of transaction $t$.  %
If all outputs are identified as self-churn, $\valO^\sendToOthers(t) = 0$ and
the algorithm continues with the next transaction.  %
%
If $\valO^\sendToOthers(t) > 0$, the algorithm collects input values in a
vector $\InpSortedT$; they are sorted in either ascending (\ac{lifo}) or
descending (\ac{fifo}) order \wrt to the timestamp when the UTXOs were
generated.  %
Then, looping over inputs $i$, input values $\valI(i)$ are added to
$\MCircMPWl(t)$ if they meet the core condition (compare line 17) introduced
in line 6 of \refalgo{code_mcirc_wb}.  %
\footnote{Summands $\MCircMPWl(t)$ can be interpreted as money drawn into
  effective circulation by transaction $t$.} %
However, one additional condition applies: If the last added input would
increase the summand $\MCircMPWl(t)$ beyond the value of outputs sent to
third parties $\valO^\sendToOthers(t)$, we only add up to the latter
amount.  %
% This effectively adds only the necessary fraction of the inputs of
% transaction $t$, generated before window $w$ or from coinbase transactions,
% that were required to match $\valO^\sendToOthers(t)$.  %

  % The components $\MCircMPWl(t)$ are consecutively summing all transaction
  % values to calculate money in circulation $\MCircMPWl$ for the given period
  % $p$ with respect to time window $\wndw$ as well as the applied sorting type.

% \par While our algorithms theoretically might be applied to any window length $\wndwLength$, in practice they lack performance for large windows. %
% Additional methods of optimization might be needed, to apply the method windows larger then 30 days. %


\subsection{Manipulation potential}
\label{sec:cc_money_seg:manipul}% 

One important concern with any measure of economic activity asks, how
amenable is it to manipulation?  In the case of velocity, the question is
specifically if the measure can be inflated by a single agent (or a small
group) of limited means.  After all, one proxy variable for velocity has been
designed as a manipulation-proof alternative to turnover (see
\refsec{results:sub:approx_crypto:subsub:bdd}).  How easy would it be to
create fake velocity in order to inflate our measures?

Indeed, no direct technical impossibility prevents generating transactions
affecting Equations~(\ref{equ:VCircWbEstPWl})--(\ref{equ:VCircMlEstPWl}).
Nonetheless, there exist reasonably tight limits to the manipulation
potential of our measures, in particular compared to trading volumes on
exchanges.

First and foremost, calculating our measure on on-chain transactions puts an upper
limit to fake transfers.  First, a fake transaction needs to be committed to
the blockchain, and thus can be repeated only once per block time.  As long
as the manipulator is not the miner confirming the next block (a random and
unlikely outcome even for large pools), the on-chain settlement is also
costly, incurring fees.

Second, the manipulator must fully fund her fake transactions: she cannot
send more than she owns once per block.  The question then turns to how she
should structure the fake transfers.  Send the entire amount in a single
transaction, or split it up into as many as can fit into a block?  In the
latter case, the fees rise.  In the former, should she minimize or maximize
the number of outputs?  If she maximizes, the fees rise again, and she
quickly fragments her wealth across a diverging multitude of wallets,
reducing the funds available in each for the next round (block) of
manipulation.

In this context it is critical that we follow the literature in clustering
user addresses: As long as the clustering works, the manipulator cannot
combine her funds ever again, lest she would be deconspired as a single agent
and all her fake transactions disregarded.  It follows that her strategy
minimizes the number of outputs, generating a chain of fresh addresses across
which a large sum, tied up in the manipulation, traverses indefinitely to
make-believe ``newcomers.''  This, however, matches peeling transactions,
which we also exclude.  She would thus need to break frequently enough in
order to escape this classification, slowing her manipulation further.

% More importantly, this strategy appears highly specific to game a measure
% which we only introduce in this paper.  Moreover, it is also straightforward
% to detect and exclude in the future.

% In sum, we are unaware of why anyone should want to pay to inflate our
% measures (unlike exchange turnover), who has the cryptocurrency funds and
% willingness and expertise to conduct such a technically elaborate
% manipulation.

In sum, manipulation attempts are both technically elaborate and relatively
straightforward to detect and exclude in the future. %
We thus do not consider manipulation a serious concern for our current
results, nor for our measures.

% 
%%% Local Variables:
%%% mode: latex
%%% TeX-master: "../main"
%%% End:
%